\documentclass[a4paper,10pt]{article}

\usepackage{algorithm}
\usepackage{algpseudocode}
\usepackage{ngerman}
\usepackage{amsmath}

\DeclareMathOperator{\oprdiv}{div}
\DeclareMathOperator{\oprshl}{shl}
\DeclareMathOperator{\oprshr}{shr}
\algblockdefx[CASE]{Case}{EndCase}
  [1]{\textbf{case} \(#1\) \textbf{of}}
  {\textbf{end\ case}}
\algblockdefx[SELECT]{Selector}{EndSelector}
  [1]{#1\textbf{: begin}}
  {\textbf{end}}
\algblockdefx[OTHER]{Other}{EndOther}
  {\textbf{otherwise: begin}}
  {\textbf{end}}
\algblockdefx[TRY]{Try}{EndTry}
  {\textbf{try}}
  {\textbf{end\ try}}
\algblockdefx[CATCH]{Catch}{EndCatch}
  [1]{\textbf{catch} (#1)}
  {\textbf{end\ catch}}\algblockdefx[FINALLY]{Finally}{EndFinally}
  {\textbf{finally}}
  {\textbf{end\ finally}}
\algblockdefx[PARALLEL]{Para}{EndPara}
  {\textbf{parallel}}
  {\textbf{end\ parallel}}
\algblockdefx[THREAD]{Thread}{EndThread}
  [1]{\textbf{thread} #1}
  [1]{\textbf{end\ thread} #1}
\algblockdefx[DECLARATION]{Decl}{EndDecl}
  [1][]{\textbf{#1}}
  {}

\title{Structorizer LaTeX pseudocode Export of FileApiGroupTest.arrz}
\author{Kay G"urtzig}
\date{09.06.2021}

\begin{document}

\begin{algorithm}
\caption{readNumbers(3)}
\begin{algorithmic}[5]
\Function{readNumbers}{fileName, numbers, maxNumbers}
\State \Comment{ Tries to read as many integer values as possible upto maxNumbers }
\State \Comment{ from file fileName into the given array numbers. }
\State \Comment{ Returns the number of the actually read numbers. May cause an exception. }
  \Decl{Parameters:}
    \State fileName: string
    \State numbers: array\ of\ integer
    \State maxNumbers: integer
  \EndDecl
  \Decl{Result type:}
    \State integer
  \EndDecl
  \State \(nNumbers\gets\ 0\);
  \State \(fileNo\gets\ fileOpen(fileName)\);
  \If{\(fileNo\leq\ 0\)}
    \State \textbf{throw} \(\)"{}File could not be opened!"{}\(\);
  \EndIf
  \Try
    \While{\(!fileEOF(fileNo)\wedge\ nNumbers<maxNumbers\)}
      \State \(number\gets\ fileReadInt(fileNo)\);
      \State \(numbers[nNumbers]\gets\ number\);
      \State \(nNumbers\gets\ nNumbers+1\);
    \EndWhile
  \EndTry
  \Catch error
    \State \textbf{throw} \(\);
  \EndCatch %1
  \Finally
    \State \(fileClose(fileNo)\);
  \EndFinally
  \State \textbf{return} \(nNumbers\);
\EndFunction
\end{algorithmic}
\end{algorithm}


\begin{algorithm}
\caption{ComputeSum}
\begin{algorithmic}[5]
\Procedure{ComputeSum}{ }
\State \Comment{ Computes the sum and average of the numbers read from a user-specified }
\State \Comment{ text file (which might have been created via generateRandomNumberFile(4)). }
\State \Comment{  }
\State \Comment{ This program is part of an arrangement used to test group code export (issue }
\State \Comment{ \#828) with FileAPI dependency. }
\State \Comment{ The input check loop has been disabled (replaced by a simple unchecked input }
\State \Comment{ instruction) in order to test the effect of indirect FileAPI dependency (only the }
\State \Comment{ called subroutine directly requires FileAPI now). }
  \State \(fileNo\gets\ 1000\);
  \State \(\)input\((\)"{}Name/path\ of\ the\ number\ file"{}\(file\_name)\);
  \Comment{Disable this if you enable the loop below!}
  \If{\(fileNo>0\)}
    \State \(values\gets\{\}\);
    \State \(nValues\gets\ 0\);
    \Try
      \State \(nValues\gets\Call{readNumbers}{file\_name,values,1000}\);
    \EndTry
    \Catch failure
      \State \(\)print\((failure)\);
      \State \textbf{exit} \(-7\);
    \EndCatch %1
    \State \(sum\gets\ 0.0\);
    \For{\(k \gets 0\) \textbf{to} \(nValues-1\) \textbf{by} \(1\)}
      \State \(sum\gets\ sum+values[k]\);
    \EndFor
    \State \(\)print\((\)"{}sum\ =\ "{}\(,sum)\);
    \State \(\)print\((\)"{}average\ =\ "{}\(,sum/nValues)\);
  \EndIf
\EndProcedure
\end{algorithmic}
\end{algorithm}

\end{document}
