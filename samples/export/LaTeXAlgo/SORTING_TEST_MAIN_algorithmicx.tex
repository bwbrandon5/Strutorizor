\documentclass[a4paper,10pt]{article}

\usepackage{algorithm}
\usepackage{algpseudocode}
\usepackage{ngerman}
\usepackage{amsmath}

\DeclareMathOperator{\oprdiv}{div}
\DeclareMathOperator{\oprshl}{shl}
\DeclareMathOperator{\oprshr}{shr}
\algblockdefx[CASE]{Case}{EndCase}
  [1]{\textbf{case} \(#1\) \textbf{of}}
  {\textbf{end\ case}}
\algblockdefx[SELECT]{Selector}{EndSelector}
  [1]{#1\textbf{: begin}}
  {\textbf{end}}
\algblockdefx[OTHER]{Other}{EndOther}
  {\textbf{otherwise: begin}}
  {\textbf{end}}
\algblockdefx[TRY]{Try}{EndTry}
  {\textbf{try}}
  {\textbf{end\ try}}
\algblockdefx[CATCH]{Catch}{EndCatch}
  [1]{\textbf{catch} (#1)}
  {\textbf{end\ catch}}\algblockdefx[FINALLY]{Finally}{EndFinally}
  {\textbf{finally}}
  {\textbf{end\ finally}}
\algblockdefx[PARALLEL]{Para}{EndPara}
  {\textbf{parallel}}
  {\textbf{end\ parallel}}
\algblockdefx[THREAD]{Thread}{EndThread}
  [1]{\textbf{thread} #1}
  [1]{\textbf{end\ thread} #1}
\algblockdefx[DECLARATION]{Decl}{EndDecl}
  [1][]{\textbf{#1}}
  {}

\title{Structorizer LaTeX pseudocode Export of SORTING\_TEST.arrz}
\author{Kay G"urtzig}
\date{09.06.2021}

\begin{document}

\begin{algorithm}
\caption{bubbleSort(1)}
\begin{algorithmic}[5]
\Procedure{bubbleSort}{values}
\State \Comment{ Implements the well-known BubbleSort algorithm. }
\State \Comment{ Compares neigbouring elements and swaps them in case of an inversion. }
\State \Comment{ Repeats this while inversions have been found. After every }
\State \Comment{ loop passage at least one element (the largest one out of the }
\State \Comment{ processed subrange) finds its final place at the end of the }
\State \Comment{ subrange. }
  \Decl{Parameters:}
    \State values: ?
  \EndDecl
  \State \(ende\gets\ length(values)-2\)
  \Repeat
    \State \(posSwapped\gets-1\)
    \Comment{The index of the most recent swapping (-1 means no swapping done).}
    \For{\(i \gets 0\) \textbf{to} \(ende\) \textbf{by} \(1\)}
      \If{\(values[i]>values[i+1]\)}
        \State \(temp\gets\ values[i]\)
        \State \(values[i]\gets\ values[i+1]\)
        \State \(values[i+1]\gets\ temp\)
        \State \(posSwapped\gets\ i\)
      \EndIf
    \EndFor
    \State \(ende\gets\ posSwapped-1\)
  \Until{\(posSwapped<0\)}
\EndProcedure
\end{algorithmic}
\end{algorithm}


\begin{algorithm}
\caption{maxHeapify(3)}
\begin{algorithmic}[5]
\Procedure{maxHeapify}{heap, i, range}
\State \Comment{ Given a max-heap '{}heap\textasciiacute{} with element at index '{}i\textasciiacute{} possibly }
\State \Comment{ violating the heap property wrt. its subtree upto and including }
\State \Comment{ index range-1, restores heap property in the subtree at index i }
\State \Comment{ again. }
  \Decl{Parameters:}
    \State heap: ?
    \State i: ?
    \State range: ?
  \EndDecl
  \State \(right\gets(i+1)*2\)
  \Comment{Indices of left and right child of node i}
  \State \(left\gets\ right-1\)
  \State \(max\gets\ i\)
  \Comment{Index of the (local) maximum}
  \If{\(left<range\wedge\ heap[left]>heap[i]\)}
    \State \(max\gets\ left\)
  \EndIf
  \If{\(right<range\wedge\ heap[right]>heap[max]\)}
    \State \(max\gets\ right\)
  \EndIf
  \If{\(max\neq\ i\)}
    \State \(temp\gets\ heap[i]\)
    \State \(heap[i]\gets\ heap[max]\)
    \State \(heap[max]\gets\ temp\)
    \State \(\Call{maxHeapify}{heap,max,range}\)
  \EndIf
\EndProcedure
\end{algorithmic}
\end{algorithm}


\begin{algorithm}
\caption{partition(4)}
\begin{algorithmic}[5]
\Function{partition}{values, start, stop, p}
\State \Comment{ Partitions array '{}values\textasciiacute{} between indices '{}start\textasciiacute{} und '{}stop\textasciiacute{}-1 with }
\State \Comment{ respect to the pivot element initially at index '{}p\textasciiacute{} into smaller }
\State \Comment{ and greater elements. }
\State \Comment{ Returns the new (and final) index of the pivot element (which }
\State \Comment{ separates the sequence of smaller elements from the sequence }
\State \Comment{ of greater elements). }
\State \Comment{ This is not the most efficient algorithm (about half the swapping }
\State \Comment{ might still be avoided) but it is pretty clear. }
  \Decl{Parameters:}
    \State values: ?
    \State start: ?
    \State stop: ?
    \State p: ?
  \EndDecl
  \Decl{Result type:}
    \State int
  \EndDecl
  \State \(pivot\gets\ values[p]\)
  \Comment{Cache the pivot element}
  \State \(values[p]\gets\ values[start]\)
  \Comment{Exchange the pivot element with the start element}
  \State \(values[start]\gets\ pivot\)
  \State \(p\gets\ start\)
  \State \(start\gets\ start+1\)
  \Comment{Beginning and end of the remaining undiscovered range}
  \State \(stop\gets\ stop-1\)
  \State \Comment{ Still unseen elements? }
  \State \Comment{ Loop invariants: }
  \State \Comment{ 1. p = start - 1 }
  \State \Comment{ 2. pivot = values[p] }
  \State \Comment{ 3. i \textless start \textrightarrow{} values[i] ≤ pivot }
  \State \Comment{ 4. stop \textless i \textrightarrow{} pivot \textless values[i] }
  \While{\(start\leq\ stop\)}
    \State \(seen\gets\ values[start]\)
    \Comment{Fetch the first element of the undiscovered area}
    \If{\(seen\leq\ pivot\)}\Comment{Does the checked element belong to the smaller area?}
      \State \(values[p]\gets\ seen\)
      \Comment{Insert the seen element between smaller area and pivot element}
      \State \(values[start]\gets\ pivot\)
      \State \Comment{ Shift the border between lower and undicovered area, }
      \State \Comment{ update pivot position. }
      \State \(p\gets\ p+1\)
      \State \(start\gets\ start+1\)
    \Else
      \State \(values[start]\gets\ values[stop]\)
      \Comment{Insert the checked element between undiscovered and larger area}
      \State \(values[stop]\gets\ seen\)
      \State \(stop\gets\ stop-1\)
      \Comment{Shift the border between undiscovered and larger area}
    \EndIf
  \EndWhile
  \State \textbf{return} \(p\)
\EndFunction
\end{algorithmic}
\end{algorithm}


\begin{algorithm}
\caption{testSorted(1)}
\begin{algorithmic}[5]
\Function{testSorted}{numbers}
\State \Comment{ Checks whether or not the passed-in array is (ascendingly) sorted. }
  \Decl{Parameters:}
    \State numbers: ?
  \EndDecl
  \Decl{Result type:}
    \State bool
  \EndDecl
  \State \(isSorted\gets\ true\)
  \State \(i\gets\ 0\)
  \While{\(isSorted\wedge(i\leq\ length(numbers)-2)\)}\Comment{As we compare with the following element, we must stop at the penultimate index}
    \If{\(numbers[i]>numbers[i+1]\)}\Comment{Is there an inversion?}
      \State \(isSorted\gets\ false\)
    \Else
      \State \(i\gets\ i+1\)
    \EndIf
  \EndWhile
  \State \textbf{return} \(isSorted\)
\EndFunction
\end{algorithmic}
\end{algorithm}


\begin{algorithm}
\caption{buildMaxHeap(1)}
\begin{algorithmic}[5]
\Procedure{buildMaxHeap}{heap}
\State \Comment{ Runs through the array heap and converts it to a max-heap }
\State \Comment{ in a bottom-up manner, i.e. starts above the "{}leaf"{} level }
\State \Comment{ (index \textgreater= length(heap) div 2) and goes then up towards }
\State \Comment{ the root. }
  \Decl{Parameters:}
    \State heap: ?
  \EndDecl
  \State \(lgth\gets\ length(heap)\)
  \For{\(k \gets lgth\oprdiv\ 2-1\) \textbf{to} \(0\) \textbf{by} \(-1\)}
    \State \(\Call{maxHeapify}{heap,k,lgth}\)
  \EndFor
\EndProcedure
\end{algorithmic}
\end{algorithm}


\begin{algorithm}
\caption{quickSort(3)}
\begin{algorithmic}[5]
\Procedure{quickSort}{values, start, stop}
\State \Comment{ Recursively sorts a subrange of the given array '{}values\textasciiacute{}.  }
\State \Comment{ start is the first index of the subsequence to be sorted, }
\State \Comment{ stop is the index BEHIND the subsequence to be sorted. }
  \Decl{Parameters:}
    \State values: ?
    \State start: ?
    \State stop: ?
  \EndDecl
  \If{\(stop\geq\ start+2\)}\Comment{At least 2 elements? (Less don'{}t make sense.)}
    \State \Comment{ Select a pivot element, be p its index. }
    \State \Comment{ (here: randomly chosen element out of start ... stop-1) }
    \State \(p\gets\ random(stop-start)+start\)
    \State \Comment{ Partition the array into smaller and greater elements }
    \State \Comment{ Get the resulting (and final) position of the pivot element }
    \State \(p\gets\Call{partition}{values,start,stop,p}\)
    \Para\Comment{Sort subsequances separately and independently ...}
      \Thread 1
        \State \(\Call{quickSort}{values,start,p}\)
        \Comment{Sort left (lower) array part}
      \EndThread 1
      \Thread 2
        \State \(\Call{quickSort}{values,p+1,stop}\)
        \Comment{Sort right (higher) array part}
      \EndThread 2
    \EndPara
  \EndIf
\EndProcedure
\end{algorithmic}
\end{algorithm}


\begin{algorithm}
\caption{heapSort(1)}
\begin{algorithmic}[5]
\Procedure{heapSort}{values}
\State \Comment{ Sorts the array '{}values\textasciiacute{} of numbers according to he heap sort }
\State \Comment{ algorithm }
  \Decl{Parameters:}
    \State values: ?
  \EndDecl
  \State \(\Call{buildMaxHeap}{values}\)
  \State \(heapRange\gets\ length(values)\)
  \For{\(k \gets heapRange-1\) \textbf{to} \(1\) \textbf{by} \(-1\)}
    \State \(heapRange\gets\ heapRange-1\)
    \State \(maximum\gets\ values[0]\)
    \Comment{Swap the maximum value (root of the heap) to the heap end}
    \State \(values[0]\gets\ values[heapRange]\)
    \State \(values[heapRange]\gets\ maximum\)
    \State \(\Call{maxHeapify}{values,0,heapRange}\)
  \EndFor
\EndProcedure
\end{algorithmic}
\end{algorithm}


\begin{algorithm}
\caption{SORTING\_TEST\_MAIN}
\begin{algorithmic}[5]
\Procedure{SORTING\_TEST\_MAIN}{ }
\State \Comment{ Creates three equal arrays of numbers and has them sorted with different sorting algorithms }
\State \Comment{ to allow performance comparison via execution counting ("{}Collect Runtime Data"{} should }
\State \Comment{ sensibly be switched on). }
\State \Comment{ Requested input data are: Number of elements (size) and filing mode. }
  \Repeat
    \State \(\)input\((elementCount)\)
  \Until{\(elementCount\geq\ 1\)}
  \Repeat
    \State \(\)input\((\)"{}Filling:\ 1\ =\ random,\ 2\ =\ increasing,\ 3\ =\ decreasing"{}\(,modus)\)
  \Until{\(modus=1\vee\ modus=2\vee\ modus=3\)}
  \For{\(i \gets 0\) \textbf{to} \(elementCount-1\) \textbf{by} \(1\)}
    \Case{modus}
      \Selector{1}
        \State \(values1[i]\gets\ random(10000)\)
      \EndSelector
      \Selector{2}
        \State \(values1[i]\gets\ i\)
      \EndSelector
      \Selector{3}
        \State \(values1[i]\gets-i\)
      \EndSelector
    \EndCase
  \EndFor
  \For{\(i \gets 0\) \textbf{to} \(elementCount-1\) \textbf{by} \(1\)}
    \State \(values2[i]\gets\ values1[i]\)
    \State \(values3[i]\gets\ values1[i]\)
  \EndFor
  \Para
    \Thread 1
      \State \(\Call{bubbleSort}{values1}\)
    \EndThread 1
    \Thread 2
      \State \(\Call{quickSort}{values2,0,elementCount}\)
    \EndThread 2
    \Thread 3
      \State \(\Call{heapSort}{values3}\)
    \EndThread 3
  \EndPara
  \State \(ok1\gets\Call{testSorted}{values1}\)
  \State \(ok2\gets\Call{testSorted}{values2}\)
  \State \(ok3\gets\Call{testSorted}{values3}\)
  \If{\(!ok1\vee!ok2\vee!ok3\)}
    \For{\(i \gets 0\) \textbf{to} \(elementCount-1\) \textbf{by} \(1\)}
      \If{\(values1[i]\neq\ values2[i]\vee\ values1[i]\neq\ values3[i]\)}
        \State \(\)print\((\)"{}Difference\ at\ ["{}\(,i,\)"{}]:\ "{}\(,values1[i],\)"{}\ \textless-\textgreater\ "{}\(,values2[i],\)"{}\ \textless-\textgreater\ "{}\(,values3[i])\)
      \EndIf
    \EndFor
  \EndIf
  \Repeat
    \State \(\)input\((\)"{}Show\ arrays\ (yes/no)?"{}\(,show)\)
  \Until{\(show=\)"{}yes"{}\(\vee\ show=\)"{}no"{}\(\)}
  \If{\(show=\)"{}yes"{}\(\)}
    \For{\(i \gets 0\) \textbf{to} \(elementCount-1\) \textbf{by} \(1\)}
      \State \(\)print\((\)"{}["{}\(,i,\)"{}]:\textbackslash{}t"{}\(,values1[i],\)"{}\textbackslash{}t"{}\(,values2[i],\)"{}\textbackslash{}t"{}\(,values3[i])\)
    \EndFor
  \EndIf
\EndProcedure
\end{algorithmic}
\end{algorithm}

\end{document}
