\documentclass[a4paper,10pt]{article}

\usepackage{pseudocode}
\usepackage{ngerman}
\usepackage{amsmath}

\DeclareMathOperator{\oprdiv}{div}
\DeclareMathOperator{\oprshl}{shl}
\DeclareMathOperator{\oprshr}{shr}
\title{Structorizer LaTeX pseudocode Export of FileApiGroupTest.arrz}
\author{Kay G"urtzig}
\date{09.06.2021}

\begin{document}

\begin{pseudocode}{readNumbers}{fileName, numbers, maxNumbers }
\label{readNumbers}
\COMMENT{ Tries to read as many integer values as possible upto maxNumbers }\\
\COMMENT{ from file fileName into the given array numbers. }\\
\COMMENT{ Returns the number of the actually read numbers. May cause an exception. }\\
\PROCEDURE{readNumbers}{fileName, numbers, maxNumbers}
  nNumbers\gets\ 0\\
  fileNo\gets\ fileOpen(fileName)\\
  \IF fileNo\leq\ 0 \THEN
    \textbf{throw}\ \)"{}File could not be opened!"{}\(\\
  \textbf{try} \BEGIN
    \WHILE \NOT\ fileEOF(fileNo)\ \AND\ nNumbers<maxNumbers \DO
    \BEGIN
      number\gets\ fileReadInt(fileNo)\\
      numbers[nNumbers]\gets\ number\\
      nNumbers\gets\ nNumbers+1\\
    \END\\
  \END\\
  \textbf{catch}\ (error)\BEGIN
    \textbf{throw}\ \\
  \END\\
  \textbf{finally} \BEGIN
    fileClose(fileNo)\\
  \END\\
  \RETURN{nNumbers}\\
\ENDPROCEDURE
\end{pseudocode}


\begin{pseudocode}{ComputeSum}{ }
\label{ComputeSum}
\COMMENT{ Computes the sum and average of the numbers read from a user-specified }\\
\COMMENT{ text file (which might have been created via generateRandomNumberFile(4)). }\\
\COMMENT{  }\\
\COMMENT{ This program is part of an arrangement used to test group code export (issue }\\
\COMMENT{ \#828) with FileAPI dependency. }\\
\COMMENT{ The input check loop has been disabled (replaced by a simple unchecked input }\\
\COMMENT{ instruction) in order to test the effect of indirect FileAPI dependency (only the }\\
\COMMENT{ called subroutine directly requires FileAPI now). }\\
\MAIN
  fileNo\gets\ 1000\\
  \COMMENT{ Disable this if you enable the loop below! }\\
  \)input\(\)"{}Name/path\ of\ the\ number\ file"{}\(file\_name\\
  \IF fileNo>0 \THEN
  \BEGIN
    values\gets\{\}\\
    nValues\gets\ 0\\
    \textbf{try} \BEGIN
      nValues\gets\CALL{readNumbers}{file\_name,values,1000}\\
    \END\\
    \textbf{catch}\ (failure)\BEGIN
      \OUTPUT{failure}\\
      \EXIT -7\\
    \END\\
    sum\gets\ 0.0\\
    \FOR k \gets 0 \TO nValues-1  \DO
      sum\gets\ sum+values[k]\\
    \OUTPUT{\)"{}sum\ =\ "{}\(,sum}\\
    \OUTPUT{\)"{}average\ =\ "{}\(,sum/nValues}\\
  \END\\
\ENDMAIN
\end{pseudocode}

\end{document}
