\documentclass[a4paper,10pt]{article}

\usepackage{algorithm}
\usepackage{algorithmic}
\usepackage{ngerman}
\usepackage{amsmath}

\DeclareMathOperator{\oprdiv}{div}
\DeclareMathOperator{\oprshl}{shl}
\DeclareMathOperator{\oprshr}{shr}
\title{Structorizer LaTeX pseudocode Export of FileApiGroupTest.arrz}
\author{Kay G"urtzig}
\date{09.06.2021}

\begin{document}

\floatname{algorithm}{Function}
\begin{algorithm}
\caption{readNumbers(fileName, numbers, maxNumbers)}
\begin{algorithmic}[5]

\STATE \COMMENT{ Tries to read as many integer values as possible upto maxNumbers }
\STATE \COMMENT{ from file fileName into the given array numbers. }
\STATE \COMMENT{ Returns the number of the actually read numbers. May cause an exception. }
  \STATE \(nNumbers\gets\ 0\);
  \STATE \(fileNo\gets\ fileOpen(fileName)\);
  \IF{\(fileNo\leq\ 0\)}
    \STATE \textbf{throw} \(\)"{}File could not be opened!"{}\(\);
  \ENDIF
  \STATE \textbf{try}  \BODY
    \WHILE{\(\NOT\ fileEOF(fileNo)\ \AND\ nNumbers<maxNumbers\)}
      \STATE \(number\gets\ fileReadInt(fileNo)\);
      \STATE \(numbers[nNumbers]\gets\ number\);
      \STATE \(nNumbers\gets\ nNumbers+1\);
    \ENDWHILE
  \ENDBODY \STATE \textbf{end try}
  \STATE \textbf{catch} (\(error\)) \BODY
    \STATE \textbf{throw} \(\);
  \ENDBODY \STATE \textbf{end catch}
  \STATE \textbf{finally} \BODY
    \STATE \(fileClose(fileNo)\);
  \ENDBODY \STATE \textbf{end finally}
  \RETURN\(nNumbers\);

\end{algorithmic}
\end{algorithm}


\floatname{algorithm}{Program}
\begin{algorithm}
\caption{ComputeSum()}
\begin{algorithmic}[5]

\STATE \COMMENT{ Computes the sum and average of the numbers read from a user-specified }
\STATE \COMMENT{ text file (which might have been created via generateRandomNumberFile(4)). }
\STATE \COMMENT{  }
\STATE \COMMENT{ This program is part of an arrangement used to test group code export (issue }
\STATE \COMMENT{ \#828) with FileAPI dependency. }
\STATE \COMMENT{ The input check loop has been disabled (replaced by a simple unchecked input }
\STATE \COMMENT{ instruction) in order to test the effect of indirect FileAPI dependency (only the }
\STATE \COMMENT{ called subroutine directly requires FileAPI now). }
  \STATE \(fileNo\gets\ 1000\);
  \STATE\ \textbf{input}\ \(\)"{}Name/path\ of\ the\ number\ file"{}\(file\_name\);
  \COMMENT{Disable this if you enable the loop below!}
  \IF{\(fileNo>0\)}
    \STATE \(values\gets\{\}\);
    \STATE \(nValues\gets\ 0\);
    \STATE \textbf{try}  \BODY
      \STATE \(nValues\gets\ readNumbers(file\_name,values,1000)\);
    \ENDBODY \STATE \textbf{end try}
    \STATE \textbf{catch} (\(failure\)) \BODY
      \PRINT\(failure\);
      \STATE \textbf{exit} \(-7\);
    \ENDBODY \STATE \textbf{end catch}
    \STATE \(sum\gets\ 0.0\);
    \FOR{\(k \gets 0\) \TO \(nValues-1\) \textbf{by} \(1\)}
      \STATE \(sum\gets\ sum+values[k]\);
    \ENDFOR
    \PRINT\(\)"{}sum\ =\ "{}\(,sum\);
    \PRINT\(\)"{}average\ =\ "{}\(,sum/nValues\);
  \ENDIF

\end{algorithmic}
\end{algorithm}

\end{document}
