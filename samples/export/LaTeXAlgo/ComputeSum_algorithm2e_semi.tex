\documentclass[a4paper,10pt]{article}

\usepackage[inoutnumbered]{algorithm2e}
\usepackage{ngerman}
\usepackage{amsmath}

\DeclareMathOperator{\oprdiv}{div}
\DeclareMathOperator{\oprshl}{shl}
\DeclareMathOperator{\oprshr}{shr}
\SetKwInput{Input}{input}
\SetKwInput{Output}{output}
\SetKwBlock{Parallel}{parallel}{end}
\SetKwFor{Thread}{thread}{:}{end}
\SetKwBlock{Try}{try}{end}
\SetKwFor{Catch}{catch (}{)}{end}
\SetKwBlock{Finally}{finally}{end}
\SetKwProg{Prog}{Program}{ }{end}
\SetKwProg{Proc}{Procedure}{ }{end}
\SetKwProg{Func}{Function}{ }{end}
\SetKwProg{Incl}{Includable}{ }{end}

\SetKw{preLeave}{leave}
\SetKw{preExit}{exit}
\SetKw{preThrow}{throw}
\PrintSemicolon
\title{Structorizer LaTeX pseudocode Export of FileApiGroupTest.arrz}
\author{Kay G"urtzig}
\date{09.06.2021}

\begin{document}
\LinesNumbered

\begin{function}
\caption{readNumbers(fileName, numbers, maxNumbers)}
\tcc{ Tries to read as many integer values as possible upto maxNumbers }
\tcc{ from file fileName into the given array numbers. }
\tcc{ Returns the number of the actually read numbers. May cause an exception. }
\Func{\FuncSty{readNumbers(}\ArgSty{fileName, numbers, maxNumbers}\FuncSty{)}:integer}{
\KwData{\(fileName\): string}
\KwData{\(numbers\): array of integer}
\KwData{\(maxNumbers\): integer}
\KwResult{integer}
  \(nNumbers\leftarrow{}0\)\;
  \(fileNo\leftarrow{}fileOpen(fileName)\)\;
  \If{\(fileNo\leq\ 0\)}{
    \preThrow{\(\)"{}File could not be opened!"{}\(\)}\;
  }
  \Try{
    \While{\(!fileEOF(fileNo)\wedge\ nNumbers<maxNumbers\)}{
      \(number\leftarrow{}fileReadInt(fileNo)\)\;
      \(numbers[nNumbers]\leftarrow{}number\)\;
      \(nNumbers\leftarrow{}nNumbers+1\)\;
    }
  }
  \Catch{error}{
    \preThrow{\(\)}\;
  }
  \Finally{
    \(fileClose(fileNo)\)\;
  }
  \Return{\(nNumbers\)};
}
\end{function}


\begin{algorithm}
\caption{ComputeSum}
\SetKwFunction{FnreadNumbers}{readNumbers}
\tcc{ Computes the sum and average of the numbers read from a user-specified }
\tcc{ text file (which might have been created via generateRandomNumberFile(4)). }
\tcc{  }
\tcc{ This program is part of an arrangement used to test group code export (issue }
\tcc{ \#828) with FileAPI dependency. }
\tcc{ The input check loop has been disabled (replaced by a simple unchecked input }
\tcc{ instruction) in order to test the effect of indirect FileAPI dependency (only the }
\tcc{ called subroutine directly requires FileAPI now). }
\Prog{\FuncSty{ComputeSum}}{
  \(fileNo\leftarrow{}1000\)\;
  \tcc{ Disable this if you enable the loop below! }
  \Input{\(\)"{}Name/path\ of\ the\ number\ file"{}\(file\_name\)}
  \If{\(fileNo>0\)}{
    \(values\leftarrow{}\{\}\)\;
    \(nValues\leftarrow{}0\)\;
    \Try{
      \(nValues\leftarrow{}\FnreadNumbers(file\_name,values,1000)\)\;
    }
    \Catch{failure}{
      \Output{\(failure\)}
      \preExit{\(-7\)}\;
    }
    \(sum\leftarrow{}0.0\)\;
    \For{\(k\leftarrow 0\) \KwTo \(nValues-1\) \textbf{by} \(1\)}{
      \(sum\leftarrow{}sum+values[k]\)\;
    }
    \Output{\(\)"{}sum\ =\ "{}\(,sum\)}
    \Output{\(\)"{}average\ =\ "{}\(,sum/nValues\)}
  }
}
\end{algorithm}

\end{document}
