\documentclass[a4paper,10pt]{article}

\usepackage{algorithm}
\usepackage{algpseudocode}
\usepackage{ngerman}
\usepackage{amsmath}

\DeclareMathOperator{\oprdiv}{div}
\DeclareMathOperator{\oprshl}{shl}
\DeclareMathOperator{\oprshr}{shr}
\algblockdefx[CASE]{Case}{EndCase}
  [1]{\textbf{case} \(#1\) \textbf{of}}
  {\textbf{end\ case}}
\algblockdefx[SELECT]{Selector}{EndSelector}
  [1]{#1\textbf{: begin}}
  {\textbf{end}}
\algblockdefx[OTHER]{Other}{EndOther}
  {\textbf{otherwise: begin}}
  {\textbf{end}}
\algblockdefx[TRY]{Try}{EndTry}
  {\textbf{try}}
  {\textbf{end\ try}}
\algblockdefx[CATCH]{Catch}{EndCatch}
  [1]{\textbf{catch} (#1)}
  {\textbf{end\ catch}}\algblockdefx[FINALLY]{Finally}{EndFinally}
  {\textbf{finally}}
  {\textbf{end\ finally}}
\algblockdefx[PARALLEL]{Para}{EndPara}
  {\textbf{parallel}}
  {\textbf{end\ parallel}}
\algblockdefx[THREAD]{Thread}{EndThread}
  [1]{\textbf{thread} #1}
  [1]{\textbf{end\ thread} #1}
\algblockdefx[DECLARATION]{Decl}{EndDecl}
  [1][]{\textbf{#1}}
  {}

\title{Structorizer LaTeX pseudocode Export of FileApiGroupTest.arrz}
\author{Kay G"urtzig}
\date{09.06.2021}

\begin{document}

\begin{algorithm}
\caption{drawBarChart(2)}
\begin{algorithmic}[5]
\Procedure{drawBarChart}{values, nValues}
\State \Comment{ Draws a bar chart from the array "{}values"{} of size nValues. }
\State \Comment{ Turtleizer must be activated and will scale the chart into a square of }
\State \Comment{ 500 x 500 pixels }
\State \Comment{ Note: The function is not robust against empty array or totally equal values. }
  \Decl{Parameters:}
    \State values: array\ of\ double
    \State nValues: ?
  \EndDecl
  \State \(const\ xSize\gets\ 500\)
  \Comment{Used range of the Turtleizer screen}
  \State \(const\ ySize\gets\ 500\)
  \State \(kMin\gets\ 0\)
  \State \(kMax\gets\ 0\)
  \For{\(k \gets 1\) \textbf{to} \(nValues-1\) \textbf{by} \(1\)}
    \If{\(values[k]>values[kMax]\)}
      \State \(kMax\gets\ k\)
    \Else
      \If{\(values[k]<values[kMin]\)}
        \State \(kMin\gets\ k\)
      \EndIf
    \EndIf
  \EndFor
  \State \(valMin\gets\ values[kMin]\)
  \State \(valMax\gets\ values[kMax]\)
  \State \(yScale\gets\ valMax*1.0/(ySize-1)\)
  \State \(yAxis\gets\ ySize-1\)
  \If{\(valMin<0\)}
    \If{\(valMax>0\)}
      \State \(yAxis\gets\ valMax*ySize*1.0/(valMax-valMin)\)
      \State \(yScale\gets(valMax-valMin)*1.0/(ySize-1)\)
    \Else
      \State \(yAxis\gets\ 1\)
      \State \(yScale\gets\ valMin*1.0/(ySize-1)\)
    \EndIf
  \EndIf
  \State \(gotoXY(1,ySize-1)\)
  \Comment{draw coordinate axes}
  \State \(forward(ySize-1)\)
  \State \(penUp()\)
  \State \(backward(yAxis)\)
  \State \(right(90)\)
  \State \(penDown()\)
  \State \(forward(xSize-1)\)
  \State \(penUp()\)
  \State \(backward(xSize-1)\)
  \State \(stripeWidth\gets\ xSize/nValues\)
  \For{\(k \gets 0\) \textbf{to} \(nValues-1\) \textbf{by} \(1\)}
    \State \(stripeHeight\gets\ values[k]*1.0/yScale\)
    \Case{k\bmod\ 3}
      \Selector{0}
        \State \(setPenColor(255,0,0)\)
      \EndSelector
      \Selector{1}
        \State \(setPenColor(0,255,0)\)
      \EndSelector
      \Selector{2}
        \State \(setPenColor(0,0,255)\)
      \EndSelector
    \EndCase
    \State \(fd(1)\)
    \State \(left(90)\)
    \State \(penDown()\)
    \State \(fd(stripeHeight)\)
    \State \(right(90)\)
    \State \(fd(stripeWidth-1)\)
    \State \(right(90)\)
    \State \(forward(stripeHeight)\)
    \State \(left(90)\)
    \State \(penUp()\)
  \EndFor
\EndProcedure
\end{algorithmic}
\end{algorithm}


\begin{algorithm}
\caption{readNumbers(3)}
\begin{algorithmic}[5]
\Function{readNumbers}{fileName, numbers, maxNumbers}
\State \Comment{ Tries to read as many integer values as possible upto maxNumbers }
\State \Comment{ from file fileName into the given array numbers. }
\State \Comment{ Returns the number of the actually read numbers. May cause an exception. }
  \Decl{Parameters:}
    \State fileName: string
    \State numbers: array\ of\ integer
    \State maxNumbers: integer
  \EndDecl
  \Decl{Result type:}
    \State integer
  \EndDecl
  \State \(nNumbers\gets\ 0\)
  \State \(fileNo\gets\ fileOpen(fileName)\)
  \If{\(fileNo\leq\ 0\)}
    \State \textbf{throw} \(\)"{}File could not be opened!"{}\(\)
  \EndIf
  \Try
    \While{\(!fileEOF(fileNo)\wedge\ nNumbers<maxNumbers\)}
      \State \(number\gets\ fileReadInt(fileNo)\)
      \State \(numbers[nNumbers]\gets\ number\)
      \State \(nNumbers\gets\ nNumbers+1\)
    \EndWhile
  \EndTry
  \Catch error
    \State \textbf{throw} \(\)
  \EndCatch %1
  \Finally
    \State \(fileClose(fileNo)\)
  \EndFinally
  \State \textbf{return} \(nNumbers\)
\EndFunction
\end{algorithmic}
\end{algorithm}


\State \Comment{ = = = = 8< = = = = = = = = = = = = = = = = = = = = = = = = = = = = = = }


\begin{algorithm}
\caption{ComputeSum}
\begin{algorithmic}[5]
\Procedure{ComputeSum}{ }
\State \Comment{ Computes the sum and average of the numbers read from a user-specified }
\State \Comment{ text file (which might have been created via generateRandomNumberFile(4)). }
\State \Comment{  }
\State \Comment{ This program is part of an arrangement used to test group code export (issue }
\State \Comment{ \#828) with FileAPI dependency. }
\State \Comment{ The input check loop has been disabled (replaced by a simple unchecked input }
\State \Comment{ instruction) in order to test the effect of indirect FileAPI dependency (only the }
\State \Comment{ called subroutine directly requires FileAPI now). }
  \State \(fileNo\gets\ 1000\)
  \State \(\)input\((\)"{}Name/path\ of\ the\ number\ file"{}\(file\_name)\)
  \Comment{Disable this if you enable the loop below!}
  \If{\(fileNo>0\)}
    \State \(values\gets\{\}\)
    \State \(nValues\gets\ 0\)
    \Try
      \State \(nValues\gets\Call{readNumbers}{file\_name,values,1000}\)
    \EndTry
    \Catch failure
      \State \(\)print\((failure)\)
      \State \textbf{exit} \(-7\)
    \EndCatch %1
    \State \(sum\gets\ 0.0\)
    \For{\(k \gets 0\) \textbf{to} \(nValues-1\) \textbf{by} \(1\)}
      \State \(sum\gets\ sum+values[k]\)
    \EndFor
    \State \(\)print\((\)"{}sum\ =\ "{}\(,sum)\)
    \State \(\)print\((\)"{}average\ =\ "{}\(,sum/nValues)\)
  \EndIf
\EndProcedure
\end{algorithmic}
\end{algorithm}


\begin{algorithm}
\caption{DrawRandomHistogram}
\begin{algorithmic}[5]
\Procedure{DrawRandomHistogram}{ }
\State \Comment{ Reads a random number file and draws a histogram accotrding to the }
\State \Comment{ user specifications }
  \State \(fileNo\gets-10\)
  \Repeat
    \State \(\)input\((\)"{}Name/path\ of\ the\ number\ file"{}\(file\_name)\)
    \State \(fileNo\gets\ fileOpen(file\_name)\)
  \Until{\(fileNo>0\vee\ file\_name=\)"{}"{}\(\)}
  \If{\(fileNo>0\)}
    \State \(fileClose(fileNo)\)
    \State \(\)input\((\)"{}number\ of\ intervals"{}\(,nIntervals)\)
    \For{\(k \gets 0\) \textbf{to} \(nIntervals-1\) \textbf{by} \(1\)}
      \State \(count[k]\gets\ 0\)
    \EndFor
    \State \(kMaxCount\gets\ 0\)
    \Comment{Index of the most populated interval}
    \State \(numberArray\gets\{\}\)
    \State \(nObtained\gets\ 0\)
    \Try
      \State \(nObtained\gets\Call{readNumbers}{file\_name,numberArray,10000}\)
    \EndTry
    \Catch failure
      \State \(\)print\((failure)\)
    \EndCatch %1
    \If{\(nObtained>0\)}
      \State \(min\gets\ numberArray[0]\)
      \State \(max\gets\ numberArray[0]\)
      \For{\(i \gets 1\) \textbf{to} \(nObtained-1\) \textbf{by} \(1\)}
        \If{\(numberArray[i]<min\)}
          \State \(min\gets\ numberArray[i]\)
        \Else
          \If{\(numberArray[i]>max\)}
            \State \(max\gets\ numberArray[i]\)
          \EndIf
        \EndIf
      \EndFor
      \State \(width\gets(max-min)*1.0/nIntervals\)
      \Comment{Interval width}
      \For{\(i \gets 0\) \textbf{to} \(nObtained-1\) \textbf{by} \(1\)}
        \State \(value\gets\ numberArray[i]\)
        \State \(k\gets\ 1\)
        \While{\(k<nIntervals\wedge\ value>min+k*width\)}
          \State \(k\gets\ k+1\)
        \EndWhile
        \State \(count[k-1]\gets\ count[k-1]+1\)
        \If{\(count[k-1]>count[kMaxCount]\)}
          \State \(kMaxCount\gets\ k-1\)
        \EndIf
      \EndFor
      \State \(\Call{drawBarChart}{count,nIntervals}\)
      \State \(\)print\((\)"{}Interval\ with\ max\ count:\ "{}\(,kMaxCount,\)"{}\ ("{}\(,count[kMaxCount],\)"{})"{}\()\)
      \For{\(k \gets 0\) \textbf{to} \(nIntervals-1\) \textbf{by} \(1\)}
        \State \(\)print\((count[k],\)"{}\ numbers\ in\ interval\ "{}\(,k,\)"{}\ ("{}\(,min+k*width,\)"{}\ ...\ "{}\(,min+(k+1)*width,\)"{})"{}\()\)
      \EndFor
    \Else
      \State \(\)print\((\)"{}No\ numbers\ read."{}\()\)
    \EndIf
  \EndIf
\EndProcedure
\end{algorithmic}
\end{algorithm}

\end{document}
