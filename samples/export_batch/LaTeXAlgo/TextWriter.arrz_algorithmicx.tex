\documentclass[a4paper,10pt]{article}

\usepackage{algorithm}
\usepackage{algpseudocode}
\usepackage{ngerman}
\usepackage{amsmath}

\DeclareMathOperator{\oprdiv}{div}
\DeclareMathOperator{\oprshl}{shl}
\DeclareMathOperator{\oprshr}{shr}
\algblockdefx[CASE]{Case}{EndCase}
  [1]{\textbf{case} \(#1\) \textbf{of}}
  {\textbf{end\ case}}
\algblockdefx[SELECT]{Selector}{EndSelector}
  [1]{#1\textbf{: begin}}
  {\textbf{end}}
\algblockdefx[OTHER]{Other}{EndOther}
  {\textbf{otherwise: begin}}
  {\textbf{end}}
\algblockdefx[TRY]{Try}{EndTry}
  {\textbf{try}}
  {\textbf{end\ try}}
\algblockdefx[CATCH]{Catch}{EndCatch}
  [1]{\textbf{catch} (#1)}
  {\textbf{end\ catch}}\algblockdefx[FINALLY]{Finally}{EndFinally}
  {\textbf{finally}}
  {\textbf{end\ finally}}
\algblockdefx[PARALLEL]{Para}{EndPara}
  {\textbf{parallel}}
  {\textbf{end\ parallel}}
\algblockdefx[THREAD]{Thread}{EndThread}
  [1]{\textbf{thread} #1}
  [1]{\textbf{end\ thread} #1}
\algblockdefx[DECLARATION]{Decl}{EndDecl}
  [1][]{\textbf{#1}}
  {}

\title{Structorizer LaTeX pseudocode Export of TextWriter.arrz}
\author{Kay G"urtzig}
\date{04.10.2021}

\begin{document}

\begin{algorithm}
\caption{backward(2)}
\begin{algorithmic}[5]
\Procedure{backward}{len, color}
  \Decl{Parameters:}
    \State len: ?
    \State color: ?
  \EndDecl
  \Case{color}
    \Selector{1}
      \State \(backward(len)\)
    \EndSelector
    \Selector{2}
      \State \(backward(len)\)
    \EndSelector
    \Selector{3}
      \State \(backward(len)\)
    \EndSelector
    \Selector{4}
      \State \(backward(len)\)
    \EndSelector
    \Selector{5}
      \State \(backward(len)\)
    \EndSelector
    \Selector{6}
      \State \(backward(len)\)
    \EndSelector
    \Selector{7}
      \State \(backward(len)\)
    \EndSelector
    \Selector{8}
      \State \(backward(len)\)
    \EndSelector
    \Selector{9}
      \State \(backward(len)\)
    \EndSelector
    \Selector{10}
      \State \(backward(len)\)
    \EndSelector
  \EndCase
\EndProcedure
\end{algorithmic}
\end{algorithm}


\begin{algorithm}
\caption{blank(2)}
\begin{algorithmic}[5]
\Procedure{blank}{h, colorNo}
\State \Comment{ Draws a blank for font height h, ignoring the colorNo }
  \Decl{Parameters:}
    \State h: ?
    \State colorNo: ?
  \EndDecl
  \State \(width\gets\ h/2.0\)
  \State \(penUp()\)
  \State \(right(90)\)
  \State \(forward(width)\)
  \State \(left(90)\)
\EndProcedure
\end{algorithmic}
\end{algorithm}


\begin{algorithm}
\caption{forward(2)}
\begin{algorithmic}[5]
\Procedure{forward}{len, color}
  \Decl{Parameters:}
    \State len: ?
    \State color: ?
  \EndDecl
  \Case{color}
    \Selector{1}
      \State \(forward(len)\)
    \EndSelector
    \Selector{2}
      \State \(forward(len)\)
    \EndSelector
    \Selector{3}
      \State \(forward(len)\)
    \EndSelector
    \Selector{4}
      \State \(forward(len)\)
    \EndSelector
    \Selector{5}
      \State \(forward(len)\)
    \EndSelector
    \Selector{6}
      \State \(forward(len)\)
    \EndSelector
    \Selector{7}
      \State \(forward(len)\)
    \EndSelector
    \Selector{8}
      \State \(forward(len)\)
    \EndSelector
    \Selector{9}
      \State \(forward(len)\)
    \EndSelector
    \Selector{10}
      \State \(forward(len)\)
    \EndSelector
  \EndCase
\EndProcedure
\end{algorithmic}
\end{algorithm}


\begin{algorithm}
\caption{digit1(2)}
\begin{algorithmic}[5]
\Procedure{digit1}{h, colorNo}
\State \Comment{ Draws digit 1 in the colour specified by colorNo with font height h }
\State \Comment{ from the current turtle position. }
  \Decl{Parameters:}
    \State h: ?
    \State colorNo: ?
  \EndDecl
  \State \(penUp()\)
  \State \(forward(h/2.0)\)
  \State \(penDown()\)
  \State \(right(45)\)
  \State \(\Call{forward}{h/sqrt(2),colorNo}\)
  \State \(left(45)\)
  \State \(\Call{backward}{h,colorNo}\)
\EndProcedure
\end{algorithmic}
\end{algorithm}


\begin{algorithm}
\caption{digit4(2)}
\begin{algorithmic}[5]
\Procedure{digit4}{h, colorNo}
\State \Comment{ Draws digit 4 in the colour specified by colorNo with font height h }
\State \Comment{ from the current turtle position. }
  \Decl{Parameters:}
    \State h: ?
    \State colorNo: ?
  \EndDecl
  \State \(b\gets\ h*0.5/(sqrt(2.0)+1)\)
  \Comment{Octagon edge length}
  \State \(c\gets\ b/sqrt(2.0)\)
  \Comment{Cathetus of the corner triangle outside the octagon}
  \State \(angle\gets\ toDegrees(atan(1-2.0*c/h))\)
  \Comment{inner angle at top of the triangle}
  \State \(right(90)\)
  \State \(penUp()\)
  \State \(forward(c+b)\)
  \State \(penDown()\)
  \State \(left(90)\)
  \State \(\Call{forward}{h,colorNo}\)
  \State \(left(180-angle)\)
  \State \(\Call{forward}{sqrt(h*h/4.0+sqr(h/2.0-c)),colorNo}\)
  \State \(left(90+angle)\)
  \State \(\Call{forward}{h/2.0,colorNo}\)
  \State \(penUp()\)
  \State \(left(90)\)
  \State \(backward(h/2.0)\)
  \State \(penDown()\)
\EndProcedure
\end{algorithmic}
\end{algorithm}


\begin{algorithm}
\caption{digit7(2)}
\begin{algorithmic}[5]
\Procedure{digit7}{h, colorNo}
\State \Comment{ Draws digit 7 in the colour specified by colorNo with font height h }
\State \Comment{ from the current turtle position. }
  \Decl{Parameters:}
    \State h: ?
    \State colorNo: ?
  \EndDecl
  \State \(angle\gets\ 90+toDegrees(atan(0.5))\)
  \State \(penUp()\)
  \State \(forward(h)\)
  \State \(penDown()\)
  \State \(right(90)\)
  \State \(\Call{forward}{h/2.0,colorNo}\)
  \State \(right(angle)\)
  \State \(\Call{forward}{h*sqrt(1.25),colorNo}\)
  \State \(left(angle)\)
  \State \(penUp()\)
  \State \(forward(h/2.0)\)
  \State \(left(90)\)
  \State \(penDown()\)
\EndProcedure
\end{algorithmic}
\end{algorithm}


\begin{algorithm}
\caption{letterA(2)}
\begin{algorithmic}[5]
\Procedure{letterA}{h, colorNo}
\State \Comment{ Draws letter A in colour specified by colorNo with font height h }
\State \Comment{ from the current turtle position. }
  \Decl{Parameters:}
    \State h: ?
    \State colorNo: ?
  \EndDecl
  \State \(width\gets\ h/2.0\)
  \State \(hypo\gets\ sqrt(h*h+width*width/4.0)\)
  \State \(rotAngle\gets\ toDegrees(atan(width/2.0/h))\)
  \State \(right(rotAngle)\)
  \State \(\Call{forward}{hypo/2.0,colorNo}\)
  \State \(right(90-rotAngle)\)
  \State \(\Call{forward}{width/2.0,colorNo}\)
  \State \(penUp()\)
  \State \(backward(width/2.0)\)
  \State \(penDown()\)
  \State \(left(90-rotAngle)\)
  \State \(\Call{forward}{hypo/2.0,colorNo}\)
  \State \(left(2*rotAngle)\)
  \State \(\Call{forward}{-hypo,colorNo}\)
  \State \(right(rotAngle)\)
\EndProcedure
\end{algorithmic}
\end{algorithm}


\begin{algorithm}
\caption{letterE(2)}
\begin{algorithmic}[5]
\Procedure{letterE}{h, colorNo}
\State \Comment{ Draws letter E in colour specified by colorNo with font height h }
\State \Comment{ from the current turtle position. }
  \Decl{Parameters:}
    \State h: ?
    \State colorNo: ?
  \EndDecl
  \State \(width\gets\ h/2.0\)
  \State \(\Call{forward}{h,colorNo}\)
  \State \(right(90)\)
  \State \(\Call{forward}{width,colorNo}\)
  \State \(right(90)\)
  \State \(penUp()\)
  \State \(forward(h/2.0)\)
  \State \(right(90)\)
  \State \(penDown()\)
  \State \(\Call{forward}{width,colorNo}\)
  \State \(left(90)\)
  \State \(penUp()\)
  \State \(forward(h/2.0)\)
  \State \(left(90)\)
  \State \(penDown()\)
  \State \(\Call{forward}{width,colorNo}\)
  \State \(left(90)\)
\EndProcedure
\end{algorithmic}
\end{algorithm}


\begin{algorithm}
\caption{letterF(2)}
\begin{algorithmic}[5]
\Procedure{letterF}{h, colorNo}
\State \Comment{ Draws letter F in colour specified by colorNo with font height h }
\State \Comment{ from the current turtle position. }
  \Decl{Parameters:}
    \State h: ?
    \State colorNo: ?
  \EndDecl
  \State \(width\gets\ h/2.0\)
  \State \(\Call{forward}{h,colorNo}\)
  \State \(right(90)\)
  \State \(\Call{forward}{width,colorNo}\)
  \State \(right(90)\)
  \State \(penUp()\)
  \State \(forward(h/2.0)\)
  \State \(right(90)\)
  \State \(penDown()\)
  \State \(\Call{forward}{width,colorNo}\)
  \State \(left(90)\)
  \State \(penUp()\)
  \State \(forward(h/2.0)\)
  \State \(left(90)\)
  \State \(forward(width)\)
  \State \(penDown()\)
  \State \(left(90)\)
\EndProcedure
\end{algorithmic}
\end{algorithm}


\begin{algorithm}
\caption{letterH(2)}
\begin{algorithmic}[5]
\Procedure{letterH}{h, colorNo}
\State \Comment{ Draws letter H in colour specified by colorNo with font height h }
\State \Comment{ from the current turtle position. }
  \Decl{Parameters:}
    \State h: ?
    \State colorNo: ?
  \EndDecl
  \State \(width\gets\ h/2.0\)
  \State \(\Call{forward}{h,colorNo}\)
  \State \(penUp()\)
  \State \(right(90)\)
  \State \(forward(width)\)
  \State \(right(90)\)
  \State \(penDown()\)
  \State \(\Call{forward}{h/2.0,colorNo}\)
  \State \(right(90)\)
  \State \(\Call{forward}{width,colorNo}\)
  \State \(penUp()\)
  \State \(backward(width)\)
  \State \(left(90)\)
  \State \(penDown()\)
  \State \(\Call{forward}{h/2.0,colorNo}\)
  \State \(left(180)\)
\EndProcedure
\end{algorithmic}
\end{algorithm}


\begin{algorithm}
\caption{letterI(2)}
\begin{algorithmic}[5]
\Procedure{letterI}{h, colorNo}
\State \Comment{ Draws letter I in colour specified by colorNo with font height h }
\State \Comment{ from the current turtle position. }
  \Decl{Parameters:}
    \State h: ?
    \State colorNo: ?
  \EndDecl
  \State \(b\gets\ h*0.5/(sqrt(2.0)+1)\)
  \Comment{Octagon edge length}
  \State \(c\gets\ b/sqrt(2.0)\)
  \Comment{Cathetus of the corner triangle outside the octagon}
  \State \(penUp()\)
  \State \(right(90)\)
  \State \(forward(c)\)
  \State \(penDown()\)
  \State \(\Call{forward}{b,colorNo}\)
  \State \(penUp()\)
  \State \(backward(b/2.0)\)
  \State \(left(90)\)
  \State \(penDown()\)
  \State \(\Call{forward}{h,colorNo}\)
  \State \(penUp()\)
  \State \(right(90)\)
  \State \(backward(b/2.0)\)
  \State \(penDown()\)
  \State \(\Call{forward}{b,colorNo}\)
  \State \(penUp()\)
  \State \(forward(b/2+c)\)
  \State \(left(90)\)
  \State \(backward(h)\)
  \State \(penDown()\)
\EndProcedure
\end{algorithmic}
\end{algorithm}


\begin{algorithm}
\caption{letterK(2)}
\begin{algorithmic}[5]
\Procedure{letterK}{h, colorNo}
\State \Comment{ Draws letter K in colour specified by colorNo with font height h }
\State \Comment{ from the current turtle position. }
  \Decl{Parameters:}
    \State h: ?
    \State colorNo: ?
  \EndDecl
  \State \(width\gets\ h/2.0\)
  \State \(diag\gets\ h/sqrt(2.0)\)
  \State \(\Call{forward}{h,colorNo}\)
  \State \(penUp()\)
  \State \(right(90)\)
  \State \(forward(width)\)
  \State \(right(135)\)
  \State \(penDown()\)
  \State \(\Call{forward}{diag,colorNo}\)
  \State \(left(90)\)
  \State \(\Call{forward}{diag,colorNo}\)
  \State \(left(135)\)
\EndProcedure
\end{algorithmic}
\end{algorithm}


\begin{algorithm}
\caption{letterL(2)}
\begin{algorithmic}[5]
\Procedure{letterL}{h, colorNo}
\State \Comment{ Draws letter L in colour specified by colorNo with font height h }
\State \Comment{ from the current turtle position. }
  \Decl{Parameters:}
    \State h: ?
    \State colorNo: ?
  \EndDecl
  \State \(width\gets\ h/2.0\)
  \State \(\Call{forward}{h,colorNo}\)
  \State \(penUp()\)
  \State \(backward(h)\)
  \State \(right(90)\)
  \State \(penDown()\)
  \State \(\Call{forward}{width,colorNo}\)
  \State \(left(90)\)
\EndProcedure
\end{algorithmic}
\end{algorithm}


\begin{algorithm}
\caption{letterM(2)}
\begin{algorithmic}[5]
\Procedure{letterM}{h, colorNo}
\State \Comment{ Draws letter M in colour specified by colorNo with font height h }
\State \Comment{ from the current turtle position. }
  \Decl{Parameters:}
    \State h: ?
    \State colorNo: ?
  \EndDecl
  \State \(width\gets\ h/2.0\)
  \State \(hypo\gets\ sqrt(width*width+h*h)/2.0\)
  \State \(rotAngle\gets\ toDegrees(atan(width/h))\)
  \State \(\Call{forward}{h,colorNo}\)
  \State \(left(rotAngle)\)
  \State \(\Call{forward}{-hypo,colorNo}\)
  \State \(right(2*rotAngle)\)
  \State \(\Call{forward}{hypo,colorNo}\)
  \State \(left(rotAngle)\)
  \State \(\Call{forward}{-h,colorNo}\)
\EndProcedure
\end{algorithmic}
\end{algorithm}


\begin{algorithm}
\caption{letterN(2)}
\begin{algorithmic}[5]
\Procedure{letterN}{h, colorNo}
\State \Comment{ Draws letter N in colour specified by colorNo with font height h }
\State \Comment{ from the current turtle position. }
  \Decl{Parameters:}
    \State h: ?
    \State colorNo: ?
  \EndDecl
  \State \(width\gets\ h/2.0\)
  \State \(hypo\gets\ sqrt(width*width+h*h)\)
  \State \(rotAngle\gets\ toDegrees(atan(width/h))\)
  \State \(\Call{forward}{h,colorNo}\)
  \State \(left(rotAngle)\)
  \State \(\Call{forward}{-hypo,colorNo}\)
  \State \(right(rotAngle)\)
  \State \(\Call{forward}{h,colorNo}\)
  \State \(penUp()\)
  \State \(backward(h)\)
  \State \(penDown()\)
\EndProcedure
\end{algorithmic}
\end{algorithm}


\begin{algorithm}
\caption{letterT(2)}
\begin{algorithmic}[5]
\Procedure{letterT}{h, colorNo}
\State \Comment{ Draws letter T in colour specified by colorNo with font height h }
\State \Comment{ from the current turtle position. }
  \Decl{Parameters:}
    \State h: ?
    \State colorNo: ?
  \EndDecl
  \State \(width\gets\ h/2.0\)
  \State \(penUp()\)
  \State \(forward(h)\)
  \State \(penDown()\)
  \State \(right(90)\)
  \State \(\Call{forward}{width,colorNo}\)
  \State \(penUp()\)
  \State \(backward(width/2.0)\)
  \State \(penDown()\)
  \State \(right(90)\)
  \State \(\Call{forward}{h,colorNo}\)
  \State \(left(90)\)
  \State \(penUp()\)
  \State \(forward(width/2.0)\)
  \State \(penDown()\)
  \State \(left(90)\)
\EndProcedure
\end{algorithmic}
\end{algorithm}


\begin{algorithm}
\caption{letterV(2)}
\begin{algorithmic}[5]
\Procedure{letterV}{h, colorNo}
\State \Comment{ Draws letter V in colour specified by colorNo with font height h }
\State \Comment{ from the current turtle position. }
  \Decl{Parameters:}
    \State h: ?
    \State colorNo: ?
  \EndDecl
  \State \(width\gets\ h/2.0\)
  \State \(hypo\gets\ sqrt(h*h+width*width/4.0)\)
  \State \(rotAngle\gets\ toDegrees(atan(width/2.0/h))\)
  \State \(penUp()\)
  \State \(forward(h)\)
  \State \(left(rotAngle)\)
  \State \(penDown()\)
  \State \(\Call{forward}{-hypo,colorNo}\)
  \State \(right(2*rotAngle)\)
  \State \(\Call{forward}{hypo,colorNo}\)
  \State \(penUp()\)
  \State \(left(rotAngle)\)
  \State \(backward(h)\)
  \State \(penDown()\)
\EndProcedure
\end{algorithmic}
\end{algorithm}


\begin{algorithm}
\caption{letterW(2)}
\begin{algorithmic}[5]
\Procedure{letterW}{h, colorNo}
\State \Comment{ Draws letter W in colour specified by colorNo with font height h }
\State \Comment{ from the current turtle position. }
  \Decl{Parameters:}
    \State h: ?
    \State colorNo: ?
  \EndDecl
  \State \(width\gets\ h/2.0\)
  \State \(width\_3\gets\ width/3.0\)
  \State \(hypo\gets\ sqrt(width\_3*width\_3+h*h)\)
  \State \(rotAngle\gets\ toDegrees(atan(width\_3/h))\)
  \State \(penUp()\)
  \State \(forward(h)\)
  \State \(left(rotAngle)\)
  \State \(penDown()\)
  \State \(\Call{forward}{-hypo,colorNo}\)
  \State \(right(2*rotAngle)\)
  \State \(\Call{forward}{hypo,colorNo}\)
  \State \(penUp()\)
  \State \(left(90+rotAngle)\)
  \State \(forward(width\_3)\)
  \State \(right(90-rotAngle)\)
  \State \(penDown()\)
  \State \(\Call{forward}{-hypo,colorNo}\)
  \State \(right(2*rotAngle)\)
  \State \(\Call{forward}{hypo,colorNo}\)
  \State \(penUp()\)
  \State \(left(rotAngle)\)
  \State \(backward(h)\)
  \State \(penDown()\)
\EndProcedure
\end{algorithmic}
\end{algorithm}


\begin{algorithm}
\caption{letterX(2)}
\begin{algorithmic}[5]
\Procedure{letterX}{h, colorNo}
\State \Comment{ Draws letter X in colour specified by colorNo with font height h }
\State \Comment{ from the current turtle position. }
  \Decl{Parameters:}
    \State h: ?
    \State colorNo: ?
  \EndDecl
  \State \(width\gets\ h/2.0\)
  \State \(hypo\gets\ sqrt(width*width+h*h)\)
  \State \(rotAngle\gets\ toDegrees(atan(width/h))\)
  \State \(right(rotAngle)\)
  \State \(\Call{forward}{hypo,colorNo}\)
  \State \(penUp()\)
  \State \(left(90+rotAngle)\)
  \State \(forward(width)\)
  \State \(right(90-rotAngle)\)
  \State \(penDown()\)
  \State \(\Call{forward}{-hypo,colorNo}\)
  \State \(right(rotAngle)\)
\EndProcedure
\end{algorithmic}
\end{algorithm}


\begin{algorithm}
\caption{letterY(2)}
\begin{algorithmic}[5]
\Procedure{letterY}{h, colorNo}
\State \Comment{ Draws letter Y in colour specified by colorNo with font height h }
\State \Comment{ from the current turtle position. }
  \Decl{Parameters:}
    \State h: ?
    \State colorNo: ?
  \EndDecl
  \State \(width\gets\ h/2.0\)
  \State \(hypo\gets\ sqrt(width*width+h*h)/2.0\)
  \State \(rotAngle\gets\ toDegrees(atan(width/h))\)
  \State \(penUp()\)
  \State \(forward(h)\)
  \State \(left(rotAngle)\)
  \State \(penDown()\)
  \State \(\Call{forward}{-hypo,colorNo}\)
  \State \(right(rotAngle)\)
  \State \(penUp()\)
  \State \(backward(h/2.0)\)
  \State \(penDown()\)
  \State \(\Call{forward}{h/2.0,colorNo}\)
  \State \(right(rotAngle)\)
  \State \(\Call{forward}{hypo,colorNo}\)
  \State \(left(rotAngle)\)
  \State \(penUp()\)
  \State \(backward(h)\)
  \State \(penDown()\)
\EndProcedure
\end{algorithmic}
\end{algorithm}


\begin{algorithm}
\caption{letterZ(2)}
\begin{algorithmic}[5]
\Procedure{letterZ}{h, colorNo}
\State \Comment{ Draws letter Z in colour specified by colorNo with font height h }
\State \Comment{ from the current turtle position. }
  \Decl{Parameters:}
    \State h: ?
    \State colorNo: ?
  \EndDecl
  \State \(width\gets\ h/2.0\)
  \State \(hypo\gets\ sqrt(width*width+h*h)\)
  \State \(rotAngle\gets\ toDegrees(atan(width/h))\)
  \State \(penUp()\)
  \State \(forward(h)\)
  \State \(right(90)\)
  \State \(penDown()\)
  \State \(\Call{forward}{width,colorNo}\)
  \State \(left(90-rotAngle)\)
  \State \(\Call{forward}{-hypo,colorNo}\)
  \State \(right(90-rotAngle)\)
  \State \(\Call{forward}{width,colorNo}\)
  \State \(left(90)\)
\EndProcedure
\end{algorithmic}
\end{algorithm}


\begin{algorithm}
\caption{polygonPart(5)}
\begin{algorithmic}[5]
\Procedure{polygonPart}{a, n, ctrclkws, nEdges, color}
\State \Comment{ Draws nEdges edges of a regular n-polygon with edge length a }
\State \Comment{ counter-clockwise, if ctrclkws is true, or clockwise if ctrclkws is false. }
  \Decl{Parameters:}
    \State a: double
    \State n: integer
    \State ctrclkws: boolean
    \State nEdges: integer
    \State color: int
  \EndDecl
  \State \(rotAngle\gets\ 360.0/n\)
  \If{\(ctrclkws\)}
    \State \(rotAngle\gets-rotAngle\)
  \EndIf
  \For{\(k \gets 1\) \textbf{to} \(nEdges\) \textbf{by} \(1\)}
    \State \(right(rotAngle)\)
    \State \(\Call{forward}{a,color}\)
  \EndFor
\EndProcedure
\end{algorithmic}
\end{algorithm}


\begin{algorithm}
\caption{charDummy(2)}
\begin{algorithmic}[5]
\Procedure{charDummy}{h, colorNo}
\State \Comment{ Draws a dummy character (small centered square) with font height h and }
\State \Comment{ the colour encoded by colorNo }
  \Decl{Parameters:}
    \State h: ?
    \State colorNo: ?
  \EndDecl
  \State \(width\gets\ h/2.0\)
  \State \(b\gets\ width/(sqrt(2.0)+1)\)
  \Comment{Octagon edge length (here: edge lengzh of the square)}
  \State \(c\gets(width-b)/2.0\)
  \Comment{Cathetus of the corner triangle outside the octagon}
  \State \(d\gets\ b/sqrt(2.0)\)
  \State \(penUp()\)
  \State \(forward(h/2.0-b/2.0)\)
  \State \(right(90)\)
  \State \(forward(c)\)
  \State \(right(90)\)
  \State \(penDown()\)
  \State \(\Call{polygonPart}{b,4,true,4,colorNo}\)
  \Comment{Draws the square with edge length b}
  \State \(penUp()\)
  \State \(left(90)\)
  \State \(forward(b+c)\)
  \State \(left(90)\)
  \State \(backward(h/2.0-b/2.0)\)
  \State \(penDown()\)
\EndProcedure
\end{algorithmic}
\end{algorithm}


\begin{algorithm}
\caption{comma(2)}
\begin{algorithmic}[5]
\Procedure{comma}{h, colorNo}
\State \Comment{ Draws a comma in colour specified by colorNo with font height h }
\State \Comment{ from the current turtle position. }
  \Decl{Parameters:}
    \State h: ?
    \State colorNo: ?
  \EndDecl
  \State \(b\gets\ h*0.5/(sqrt(2.0)+1)\)
  \Comment{Octagon edge length}
  \State \(c\gets\ b/sqrt(2.0)\)
  \Comment{Cathetus of the outer corner triangle of the octagon}
  \State \(rotAngle\gets\ toDegrees(atan(0.5))\)
  \State \(hypo\gets\ c*sqrt(1.25)\)
  \State \(penUp()\)
  \State \(right(90)\)
  \State \(forward((c+b)/2.0+c)\)
  \State \(penDown()\)
  \State \Comment{ Counterclockwise draw 3 edges of a square with edge length c }
  \State \Comment{ in the colour endcoded by colorNo }
  \State \(\Call{polygonPart}{c,4,true,3,colorNo}\)
  \State \(left(90)\)
  \State \(\Call{forward}{c/2.0,colorNo}\)
  \State \(right(90)\)
  \State \(\Call{forward}{c,colorNo}\)
  \State \(left(180-rotAngle)\)
  \State \(\Call{forward}{hypo,colorNo}\)
  \State \(penUp()\)
  \State \(right(90-rotAngle)\)
  \State \(forward((c+b)/2.0)\)
  \State \(left(90)\)
  \State \(penDown()\)
\EndProcedure
\end{algorithmic}
\end{algorithm}


\begin{algorithm}
\caption{digit2(2)}
\begin{algorithmic}[5]
\Procedure{digit2}{h, colorNo}
\State \Comment{ Draws digit 2 in the colour specified by colorNo with font height h }
\State \Comment{ from the current turtle position. }
  \Decl{Parameters:}
    \State h: ?
    \State colorNo: ?
  \EndDecl
  \State \(b\gets\ h*0.5/(sqrt(2.0)+1)\)
  \Comment{Octagon edge length}
  \State \(c\gets\ b/sqrt(2.0)\)
  \Comment{Cathetus of the corner triangle outside the octagon}
  \State \(angle\gets\ toDegrees(atan(h/(h+2*c)))\)
  \State \(penUp()\)
  \State \(forward(h-c)\)
  \State \(penDown()\)
  \State \Comment{ Clockwise draw 4 edges of an octagon with edge length b }
  \State \Comment{ in the colour endcoded by colorNo }
  \State \(\Call{polygonPart}{b,8,false,4,colorNo}\)
  \State \(right(angle)\)
  \State \(\Call{forward}{h/2.0*sqrt(1+sqr(1+2*c/h)),colorNo}\)
  \State \(left(90+angle)\)
  \State \(\Call{forward}{h/2.0,colorNo}\)
  \State \(left(90)\)
\EndProcedure
\end{algorithmic}
\end{algorithm}


\begin{algorithm}
\caption{digit3(2)}
\begin{algorithmic}[5]
\Procedure{digit3}{h, colorNo}
\State \Comment{ Draws digit 3 in the colour specified by colorNo with font height h }
\State \Comment{ from the current turtle position. }
  \Decl{Parameters:}
    \State h: ?
    \State colorNo: ?
  \EndDecl
  \State \(b\gets\ h*0.5/(sqrt(2.0)+1)\)
  \Comment{Octagon edge length}
  \State \(c\gets\ b/sqrt(2.0)\)
  \Comment{Cathetus of the corner triangle outside the octagon}
  \State \(penUp()\)
  \State \(forward(c)\)
  \State \(penDown()\)
  \State \(right(180)\)
  \State \Comment{ Counterclockwise draw 6 edges of an octagon with edge length b }
  \State \Comment{ in the colour endcoded by colorNo }
  \State \(\Call{polygonPart}{b,8,true,6,colorNo}\)
  \State \(penUp()\)
  \State \(left(180)\)
  \State \(forward(b)\)
  \State \(penDown()\)
  \State \Comment{ Counterclockwise draw 5 edges of an octagon with edge length b }
  \State \Comment{ in the colour endcoded by colorNo }
  \State \(\Call{polygonPart}{b,8,true,5,colorNo}\)
  \State \(penUp()\)
  \State \(left(45)\)
  \State \(forward(h-c)\)
  \State \(left(90)\)
  \State \(forward(h/2.0)\)
  \State \(left(90)\)
  \State \(penDown()\)
\EndProcedure
\end{algorithmic}
\end{algorithm}


\begin{algorithm}
\caption{digit5(2)}
\begin{algorithmic}[5]
\Procedure{digit5}{h, colorNo}
\State \Comment{ Draws digit 5 in the colour specified by colorNo with font height h }
\State \Comment{ from the current turtle position. }
  \Decl{Parameters:}
    \State h: ?
    \State colorNo: ?
  \EndDecl
  \State \(b\gets\ h*0.5/(sqrt(2.0)+1)\)
  \Comment{Octagon edge length}
  \State \(c\gets\ b/sqrt(2.0)\)
  \Comment{Cathetus of the corner triangle outside the octagon}
  \State \(penUp()\)
  \State \(forward(c)\)
  \State \(penDown()\)
  \State \(right(180)\)
  \State \Comment{ Counterclockwise draw 6 edges of an octagon with edge length b }
  \State \Comment{ in the colour endcoded by colorNo }
  \State \(\Call{polygonPart}{b,8,true,6,colorNo}\)
  \State \(\Call{forward}{c,colorNo}\)
  \State \Comment{ Clockwise draw 2 edges of an octagon with edge length b }
  \State \Comment{ in the colour endcoded by colorNo }
  \State \(\Call{polygonPart}{h/2.0,4,false,2,colorNo}\)
  \State \(penUp()\)
  \State \(left(90)\)
  \State \(backward(h)\)
  \State \(penDown()\)
\EndProcedure
\end{algorithmic}
\end{algorithm}


\begin{algorithm}
\caption{digit6(2)}
\begin{algorithmic}[5]
\Procedure{digit6}{h, colorNo}
\State \Comment{ Draws digit 6 in the colour specified by colorNo with font height h }
\State \Comment{ from the current turtle position. }
  \Decl{Parameters:}
    \State h: ?
    \State colorNo: ?
  \EndDecl
  \State \(b\gets\ h*0.5/(sqrt(2.0)+1)\)
  \Comment{Octagon edge length}
  \State \(c\gets\ b/sqrt(2.0)\)
  \Comment{Cathetus of the corner triangle outside the octagon}
  \State \(penUp()\)
  \State \(forward(c)\)
  \State \(penDown()\)
  \State \(right(180)\)
  \State \Comment{ Counterclockwise draw all 8 edges of an octagon with edge length b }
  \State \Comment{ in the colour endcoded by colorNo }
  \State \(\Call{polygonPart}{b,8,true,8,colorNo}\)
  \State \(penUp()\)
  \State \(left(180)\)
  \State \(forward(b)\)
  \State \(penDown()\)
  \State \(\Call{forward}{2*c+b,colorNo}\)
  \State \Comment{ Clockwise draw 3 edges of an octagon with edge length b }
  \State \Comment{ in the colour endcoded by colorNo }
  \State \(\Call{polygonPart}{b,8,false,3,colorNo}\)
  \State \(penUp()\)
  \State \(left(135)\)
  \State \(backward(h-c)\)
  \State \(penDown()\)
\EndProcedure
\end{algorithmic}
\end{algorithm}


\begin{algorithm}
\caption{digit8(2)}
\begin{algorithmic}[5]
\Procedure{digit8}{h, colorNo}
\State \Comment{ Draws digit 8 in the colour specified by colorNo with font height h }
\State \Comment{ from the current turtle position. }
  \Decl{Parameters:}
    \State h: ?
    \State colorNo: ?
  \EndDecl
  \State \(b\gets\ h*0.5/(sqrt(2.0)+1)\)
  \Comment{Octagon edge length}
  \State \(c\gets\ b/sqrt(2.0)\)
  \Comment{Cathetus of the corner triangle outside the octagon}
  \State \(penUp()\)
  \State \(forward(c)\)
  \State \(penDown()\)
  \State \(right(180)\)
  \State \Comment{ Counterclockwise draw all 8 edges of an octagon with edge length b }
  \State \Comment{ in the colour endcoded by colorNo }
  \State \(\Call{polygonPart}{b,8,true,8,colorNo}\)
  \State \(penUp()\)
  \State \(left(180)\)
  \State \(forward(b)\)
  \State \(right(45)\)
  \State \(forward(b)\)
  \State \(left(135)\)
  \State \(penDown()\)
  \State \Comment{ Clockwise draw 7 edges of an octagon with edge length b }
  \State \Comment{ in the colour endcoded by colorNo }
  \State \(\Call{polygonPart}{b,8,false,7,colorNo}\)
  \State \(penUp()\)
  \State \(left(45)\)
  \State \(forward(h/2.0)\)
  \State \(left(90)\)
  \State \(forward(c)\)
  \State \(left(90)\)
  \State \(penDown()\)
\EndProcedure
\end{algorithmic}
\end{algorithm}


\begin{algorithm}
\caption{digit9(2)}
\begin{algorithmic}[5]
\Procedure{digit9}{h, colorNo}
\State \Comment{ Draws digit 9 in the colour specified by colorNo with font height h }
\State \Comment{ from the current turtle position. }
  \Decl{Parameters:}
    \State h: ?
    \State colorNo: ?
  \EndDecl
  \State \(b\gets\ h*0.5/(sqrt(2.0)+1)\)
  \Comment{Octagon edge length}
  \State \(c\gets\ b/sqrt(2.0)\)
  \Comment{Cathetus of the corner triangle outside the octagon}
  \State \(penUp()\)
  \State \(forward(c)\)
  \State \(penDown()\)
  \State \(right(180)\)
  \State \Comment{ Counterclockwise draw 4 edges of an octagon with edge length b }
  \State \Comment{ in the colour endcoded by colorNo }
  \State \(\Call{polygonPart}{b,8,true,4,colorNo}\)
  \State \(\Call{forward}{2*c+b,colorNo}\)
  \State \Comment{ Counterclockwise draw 7 edges of an octagon with edge length b }
  \State \Comment{ in the colour endcoded by colorNo }
  \State \(\Call{polygonPart}{b,8,true,7,colorNo}\)
  \State \(penUp()\)
  \State \(left(45)\)
  \State \(backward(h/2.0+c)\)
  \State \(penDown()\)
\EndProcedure
\end{algorithmic}
\end{algorithm}


\begin{algorithm}
\caption{exclMk(2)}
\begin{algorithmic}[5]
\Procedure{exclMk}{h, colorNo}
\State \Comment{ Draws an exclamation mark in the colour encoded by colorNo with font height h }
\State \Comment{ from the current turtle position. }
  \Decl{Parameters:}
    \State h: ?
    \State colorNo: ?
  \EndDecl
  \State \(b\gets\ h*0.5/(sqrt(2.0)+1)\)
  \Comment{Octagon edge length}
  \State \(c\gets\ b/sqrt(2.0)\)
  \Comment{Cathetus of the outer corner triangle of the octagon}
  \State \(width\gets\ h/2.0\)
  \State \(length1\gets\ h-(b+c)/2.0\)
  \State \(length2\gets\ length1-2*c\)
  \State \(hypo\gets\ sqrt(width*width/16.0+length2*length2)\)
  \State \(rotAngle\gets\ 45\)
  \Comment{360\textdegree/8}
  \State \(rotAngle2\gets\ toDegrees(atan(width/4.0/length2))\)
  \State \(penUp()\)
  \State \(forward(length1)\)
  \State \(right(90)\)
  \State \(forward(width/2.0)\)
  \State \(left(90+rotAngle)\)
  \State \(penDown()\)
  \State \Comment{ Clockwise draw 5 edges of an octagon with edge length b/2 }
  \State \Comment{ in the colour endcoded by colorNo }
  \State \(\Call{polygonPart}{b/2.0,8,false,5,colorNo}\)
  \State \(right(rotAngle2)\)
  \State \(\Call{forward}{hypo,colorNo}\)
  \State \(left(2*rotAngle2)\)
  \State \(\Call{forward}{-hypo,colorNo}\)
  \State \(penUp()\)
  \State \(forward(hypo)\)
  \State \(right(rotAngle2)\)
  \State \(forward(c)\)
  \State \(left(90)\)
  \State \(forward(c/2.0)\)
  \State \(penDown()\)
  \State \Comment{ Counterclockwise draw all 4 edges of a squarfe with edge length c }
  \State \Comment{ in the colour endcoded by colorNo }
  \State \(\Call{polygonPart}{c,4,false,4,colorNo}\)
  \State \(penUp()\)
  \State \(forward((c+b)/2.0)\)
  \State \(left(90)\)
  \State \(backward(c)\)
  \State \(penDown()\)
\EndProcedure
\end{algorithmic}
\end{algorithm}


\begin{algorithm}
\caption{fullSt(2)}
\begin{algorithmic}[5]
\Procedure{fullSt}{h, colorNo}
\State \Comment{ Draws a full stop in colour specified by colorNo with font height h }
\State \Comment{ from the current turtle position. }
  \Decl{Parameters:}
    \State h: ?
    \State colorNo: ?
  \EndDecl
  \State \(b\gets\ h*0.5/(sqrt(2.0)+1)\)
  \Comment{Octagon edge length}
  \State \(c\gets\ b/sqrt(2.0)\)
  \Comment{Cathetus of the outer corner triangle of the octagon}
  \State \(penUp()\)
  \State \(right(90)\)
  \State \(forward((c+b)/2.0+c)\)
  \State \(penDown()\)
  \State \Comment{ Counterclockwise draw all 4 edges of a squarfe with edge length c }
  \State \Comment{ in the colour endcoded by colorNo }
  \State \(\Call{polygonPart}{c,4,true,4,colorNo}\)
  \State \(penUp()\)
  \State \(forward((c+b)/2.0)\)
  \State \(left(90)\)
  \State \(penDown()\)
\EndProcedure
\end{algorithmic}
\end{algorithm}


\begin{algorithm}
\caption{letterAe(2)}
\begin{algorithmic}[5]
\Procedure{letterAe}{h, colorNo}
\State \Comment{ Draws letter "A in colour specified by colorNo with font height h }
\State \Comment{ from the current turtle position. }
  \Decl{Parameters:}
    \State h: ?
    \State colorNo: ?
  \EndDecl
  \State \(penUp()\)
  \State \(forward(h)\)
  \State \(penDown()\)
  \State \Comment{ Clockwise draw all 4 edges of a square with edge length h/16 }
  \State \Comment{ in the colour endcoded by colorNo }
  \State \(\Call{polygonPart}{max(h/16.0,1),4,false,4,colorNo}\)
  \State \(right(90)\)
  \State \(penUp()\)
  \State \(forward(h/2.0)\)
  \State \(penDown()\)
  \State \Comment{ Clockwise draw all 4 edges of a square with edge length h/16 }
  \State \Comment{ in the colour endcoded by colorNo }
  \State \(\Call{polygonPart}{max(h/16.0,1),4,false,4,colorNo}\)
  \State \(right(90)\)
  \State \(penUp()\)
  \State \(forward(h)\)
  \State \(right(90)\)
  \State \(forward(h/2.0)\)
  \State \(penDown()\)
  \State \(right(90)\)
  \State \(\Call{letterA}{h,colorNo}\)
\EndProcedure
\end{algorithmic}
\end{algorithm}


\begin{algorithm}
\caption{letterB(2)}
\begin{algorithmic}[5]
\Procedure{letterB}{h, colorNo}
\State \Comment{ Draws letter B in colour specified by colorNo with font height h }
\State \Comment{ from the current turtle position. }
  \Decl{Parameters:}
    \State h: ?
    \State colorNo: ?
  \EndDecl
  \State \(b\gets\ h*0.5/(sqrt(2.0)+1)\)
  \Comment{Octagon edge length}
  \State \(c\gets\ b/sqrt(2.0)\)
  \Comment{Cathetus of the outer corner triangle of the octagon}
  \State \(\Call{forward}{h,colorNo}\)
  \State \(right(90)\)
  \State \(\Call{forward}{c+b,colorNo}\)
  \State \(\Call{polygonPart}{b,8,false,4,colorNo}\)
  \Comment{Clockwise draw 4 edges of an octagon with edge length b}
  \State \(\Call{forward}{c,colorNo}\)
  \State \(penUp()\)
  \State \(left(180)\)
  \State \(forward(b+c)\)
  \State \(penDown()\)
  \State \(\Call{polygonPart}{b,8,false,4,colorNo}\)
  \Comment{Clockwise draw 4 edges of an octagon with edge length b}
  \State \(\Call{forward}{c,colorNo}\)
  \State \(penUp()\)
  \State \(left(180)\)
  \State \(forward(b+2*c)\)
  \State \(penDown()\)
  \State \(left(90)\)
\EndProcedure
\end{algorithmic}
\end{algorithm}


\begin{algorithm}
\caption{letterC(2)}
\begin{algorithmic}[5]
\Procedure{letterC}{h, colorNo}
\State \Comment{ Draws letter C in the colour encoded by colorNo with font height h }
\State \Comment{ from the current turtle position. }
  \Decl{Parameters:}
    \State h: ?
    \State colorNo: ?
  \EndDecl
  \State \(b\gets\ h*0.5/(sqrt(2.0)+1)\)
  \Comment{Octagon edge length}
  \State \(c\gets\ b/sqrt(2.0)\)
  \Comment{Cathetus of the outer triangle at the octagon corner}
  \State \(rotAngle\gets\ 45\)
  \Comment{360\textdegree/8}
  \State \(penUp()\)
  \State \(forward(c)\)
  \State \(penDown()\)
  \State \(right(180)\)
  \State \Comment{ Clockwise draws 3 edges of an octagon with edge length b in the colour }
  \State \Comment{ encoded by colorNo }
  \State \(\Call{polygonPart}{b,8,true,3,colorNo}\)
  \State \(left(rotAngle)\)
  \State \(penUp()\)
  \State \(forward(2*b+2*c)\)
  \State \(penDown()\)
  \State \Comment{ Counterclockwise draws 4 edges of an octagon with edge length b }
  \State \Comment{ iin the colour encoded by colorNo }
  \State \(\Call{polygonPart}{b,8,true,4,colorNo}\)
  \State \(\Call{forward}{b+2*c,colorNo}\)
  \State \(penUp()\)
  \State \(forward(c)\)
  \State \(left(90)\)
  \State \(\Call{forward}{b+2*c,colorNo}\)
  \State \(penDown()\)
  \State \(left(90)\)
\EndProcedure
\end{algorithmic}
\end{algorithm}


\begin{algorithm}
\caption{letterD(2)}
\begin{algorithmic}[5]
\Procedure{letterD}{h, colorNo}
\State \Comment{ Draws letter D in colour specified by colorNo with font height h }
\State \Comment{ from the current turtle position. }
  \Decl{Parameters:}
    \State h: ?
    \State colorNo: ?
  \EndDecl
  \State \(b\gets\ h*0.5/(sqrt(2.0)+1)\)
  \Comment{Octagon edge length}
  \State \(c\gets\ b/sqrt(2.0)\)
  \Comment{Cathetus of the outer corner triangle of the octagon}
  \State \(\Call{forward}{h,colorNo}\)
  \State \(right(90)\)
  \State \(\Call{forward}{c+b,colorNo}\)
  \State \Comment{ Clockwise draw 2 edges of an octagon with edge length b in the colour }
  \State \Comment{ encoded by colorNo }
  \State \(\Call{polygonPart}{b,8,false,2,colorNo}\)
  \State \(\Call{forward}{b+2*c,colorNo}\)
  \State \Comment{ Clockwise draw 2 edges of an octagon with edge length b in the colour }
  \State \Comment{ encoded by colorNo }
  \State \(\Call{polygonPart}{b,8,false,2,colorNo}\)
  \State \(\Call{forward}{c,colorNo}\)
  \State \(penUp()\)
  \State \(left(180)\)
  \State \(forward(b+2*c)\)
  \State \(penDown()\)
  \State \(left(90)\)
\EndProcedure
\end{algorithmic}
\end{algorithm}


\begin{algorithm}
\caption{letterG(2)}
\begin{algorithmic}[5]
\Procedure{letterG}{h, colorNo}
\State \Comment{ Draws letter G in colour specified by colorNo with font height h }
\State \Comment{ from the current turtle position. }
  \Decl{Parameters:}
    \State h: ?
    \State colorNo: ?
  \EndDecl
  \State \(b\gets\ h*0.5/(sqrt(2.0)+1)\)
  \Comment{Octagon edge length}
  \State \(c\gets\ b/sqrt(2.0)\)
  \Comment{Cathetus of the corner triangle outside the octagon.}
  \State \(penUp()\)
  \State \(forward(c)\)
  \State \(penDown()\)
  \State \(right(180)\)
  \State \Comment{ Counterclockwise draw 4 edges of an octagon with edge length b in }
  \State \Comment{ the colour encoded by colorNo }
  \State \(\Call{polygonPart}{b,8,true,4,colorNo}\)
  \State \(\Call{forward}{c,colorNo}\)
  \State \(left(90)\)
  \State \(\Call{forward}{b/2.0+c,colorNo}\)
  \State \(penUp()\)
  \State \(backward(b/2.0+c)\)
  \State \(right(90)\)
  \State \(forward(b+c)\)
  \State \(penDown()\)
  \State \Comment{ Counterclockwise draw 4 edges of an octagon with edge length b in }
  \State \Comment{ the colour encoded by colorNo }
  \State \(\Call{polygonPart}{b,8,true,4,colorNo}\)
  \State \(\Call{forward}{b+2*c,colorNo}\)
  \State \(penUp()\)
  \State \(forward(c)\)
  \State \(left(90)\)
  \State \(\Call{forward}{b+2*c,colorNo}\)
  \State \(penDown()\)
  \State \(left(90)\)
\EndProcedure
\end{algorithmic}
\end{algorithm}


\begin{algorithm}
\caption{letterJ(2)}
\begin{algorithmic}[5]
\Procedure{letterJ}{h, colorNo}
\State \Comment{ Draws letter J in colour encoded by colorNo with font height h }
\State \Comment{ from the current turtle position. }
  \Decl{Parameters:}
    \State h: ?
    \State colorNo: ?
  \EndDecl
  \State \(b\gets\ h*0.5/(sqrt(2.0)+1)\)
  \Comment{Octagon edge length}
  \State \(c\gets\ b/sqrt(2.0)\)
  \Comment{Cathetus of the outer corner triangle of the octagon}
  \State \(rotAngle\gets\ 45\)
  \Comment{360\textdegree/8}
  \State \(penUp()\)
  \State \(forward(c)\)
  \State \(penDown()\)
  \State \(right(180)\)
  \State \Comment{ Counterclockwise draw 3 edges of an octagon with edge length b in }
  \State \Comment{ the colour encoded by colorNo }
  \State \(\Call{polygonPart}{b,8,true,3,colorNo}\)
  \State \(left(rotAngle)\)
  \State \(\Call{forward}{h-c,colorNo}\)
  \State \(penUp()\)
  \State \(backward(h)\)
  \State \(penDown()\)
\EndProcedure
\end{algorithmic}
\end{algorithm}


\begin{algorithm}
\caption{letterO(2)}
\begin{algorithmic}[5]
\Procedure{letterO}{h, colorNo}
\State \Comment{ Draws letter O in colour specified by colorNo with font height h }
\State \Comment{ from the current turtle position. }
  \Decl{Parameters:}
    \State h: ?
    \State colorNo: ?
  \EndDecl
  \State \(b\gets\ h*0.5/(sqrt(2.0)+1)\)
  \Comment{Octagon edge length}
  \State \(c\gets\ b/sqrt(2.0)\)
  \Comment{Cathetus of the corner triangle outside the octagon}
  \State \(penUp()\)
  \State \(forward(c)\)
  \State \(penDown()\)
  \State \(right(180)\)
  \State \Comment{ Counterclockwise draw 4 edges of an octagon with edge length b }
  \State \Comment{ in the colour endcoded by colorNo }
  \State \(\Call{polygonPart}{b,8,true,4,colorNo}\)
  \State \(\Call{forward}{b+2*c,colorNo}\)
  \State \Comment{ Counterclockwise draw 4 edges of an octagon with edge length b }
  \State \Comment{ in the colour endcoded by colorNo }
  \State \(\Call{polygonPart}{b,8,true,4,colorNo}\)
  \State \(\Call{forward}{b+2*c,colorNo}\)
  \State \(penUp()\)
  \State \(forward(c)\)
  \State \(left(90)\)
  \State \(forward(b+2*c)\)
  \State \(penDown()\)
  \State \(left(90)\)
\EndProcedure
\end{algorithmic}
\end{algorithm}


\begin{algorithm}
\caption{letterP(2)}
\begin{algorithmic}[5]
\Procedure{letterP}{h, colorNo}
\State \Comment{ Draws letter P in colour specified by colorNo with font height h }
\State \Comment{ from the current turtle position. }
  \Decl{Parameters:}
    \State h: ?
    \State colorNo: ?
  \EndDecl
  \State \(b\gets\ h*0.5/(sqrt(2.0)+1)\)
  \Comment{Octagon edge length}
  \State \(c\gets\ b/sqrt(2.0)\)
  \Comment{Cathetus of the corner triangle outside the octagon}
  \State \(\Call{forward}{h,colorNo}\)
  \State \(right(90)\)
  \State \(\Call{forward}{c+b,colorNo}\)
  \State \Comment{ Clockwise draw 4 edges of an octagon with edge length b }
  \State \Comment{ in the colour endcoded by colorNo }
  \State \(\Call{polygonPart}{b,8,false,4,colorNo}\)
  \State \(\Call{forward}{c,colorNo}\)
  \State \(penUp()\)
  \State \(backward(b+2*c)\)
  \State \(left(90)\)
  \State \(forward(b+2*c)\)
  \State \(penDown()\)
  \State \(left(180)\)
\EndProcedure
\end{algorithmic}
\end{algorithm}


\begin{algorithm}
\caption{letterQ(2)}
\begin{algorithmic}[5]
\Procedure{letterQ}{h, colorNo}
\State \Comment{ Draws letter Q in colour specified by colorNo with font height h }
\State \Comment{ from the current turtle position. }
  \Decl{Parameters:}
    \State h: ?
    \State colorNo: ?
  \EndDecl
  \State \(b\gets\ h*0.5/(sqrt(2.0)+1)\)
  \Comment{Octagon edge length}
  \State \(c\gets\ b/sqrt(2.0)\)
  \Comment{Cathetus of the outer corner triangle of the octagon}
  \State \(rotAngle\gets\ 45\)
  \Comment{360\textdegree/8}
  \State \(penUp()\)
  \State \(forward(c)\)
  \State \(penDown()\)
  \State \(right(180)\)
  \State \Comment{ Counterclockwise draw 4 edges of an octagon with edge length b }
  \State \Comment{ in the colour endcoded by colorNo }
  \State \(\Call{polygonPart}{b,8,true,4,colorNo}\)
  \State \(\Call{forward}{b+2*c,colorNo}\)
  \State \Comment{ Counterclockwise draw 4 edges of an octagon with edge length b }
  \State \Comment{ in the colour endcoded by colorNo }
  \State \(\Call{polygonPart}{b,8,true,4,colorNo}\)
  \State \(\Call{forward}{b+2*c,colorNo}\)
  \State \(penUp()\)
  \State \(forward(c)\)
  \State \(left(90)\)
  \State \(forward(b+2*c)\)
  \State \(right(rotAngle)\)
  \State \(backward(b)\)
  \State \(penDown()\)
  \State \(\Call{forward}{b,colorNo}\)
  \State \(left(90+rotAngle)\)
\EndProcedure
\end{algorithmic}
\end{algorithm}


\begin{algorithm}
\caption{letterR(2)}
\begin{algorithmic}[5]
\Procedure{letterR}{h, colorNo}
\State \Comment{ Zeichnet den Buchstaben R von der Turtleposition aus }
\State \Comment{ mit Zeilenh"ohe h }
  \Decl{Parameters:}
    \State h: ?
    \State colorNo: ?
  \EndDecl
  \State \(b\gets\ h*0.5/(sqrt(2.0)+1)\)
  \Comment{Octagon edge length}
  \State \(c\gets\ b/sqrt(2.0)\)
  \Comment{Cathetus of the outer corner triangle of the octagon}
  \State \(rotAngle\gets\ 45\)
  \Comment{360\textdegree/8}
  \State \(\Call{forward}{h,colorNo}\)
  \State \(right(90)\)
  \State \(\Call{forward}{c+b,colorNo}\)
  \State \Comment{ Clockwise draw 4 edges of an octagon with edge length b }
  \State \Comment{ in the colour endcoded by colorNo }
  \State \(\Call{polygonPart}{b,8,false,4,colorNo}\)
  \State \(\Call{forward}{c,colorNo}\)
  \State \(left(90+rotAngle)\)
  \State \(\Call{forward}{sqrt(2.0)*(b+2*c),colorNo}\)
  \State \(left(90+rotAngle)\)
\EndProcedure
\end{algorithmic}
\end{algorithm}


\begin{algorithm}
\caption{letterS(2)}
\begin{algorithmic}[5]
\Procedure{letterS}{h, colorNo}
\State \Comment{ Draws letter S in colour specified by colorNo with font height h }
\State \Comment{ from the current turtle position. }
  \Decl{Parameters:}
    \State h: ?
    \State colorNo: ?
  \EndDecl
  \State \(b\gets\ h*0.5/(sqrt(2.0)+1)\)
  \Comment{Octagon edge length}
  \State \(c\gets\ b/sqrt(2.0)\)
  \Comment{Side length of the (outer) corner triangle of the octagon}
  \State \(rotAngle\gets\ 45\)
  \Comment{360\textdegree/8}
  \State \(penUp()\)
  \State \(forward(c)\)
  \State \(penDown()\)
  \State \(right(180)\)
  \State \Comment{ Counterclockwise draw 6 edges of an octagon with edge length b }
  \State \Comment{ in the colour encoded by colorNo }
  \State \(\Call{polygonPart}{b,8,true,6,colorNo}\)
  \State \Comment{ Clockwise draw 5 edges of an octagon with edge length b }
  \State \Comment{ in the colour encoded by colorNo }
  \State \(\Call{polygonPart}{b,8,false,5,colorNo}\)
  \State \(right(rotAngle)\)
  \State \(penUp()\)
  \State \(forward(2*b+3*c)\)
  \State \(penDown()\)
  \State \(left(180)\)
\EndProcedure
\end{algorithmic}
\end{algorithm}


\begin{algorithm}
\caption{letterU(2)}
\begin{algorithmic}[5]
\Procedure{letterU}{h, colorNo}
\State \Comment{ Draws letter U in colour specified by colorNo with font height h }
\State \Comment{ from the current turtle position. }
  \Decl{Parameters:}
    \State h: ?
    \State colorNo: ?
  \EndDecl
  \State \(b\gets\ h*0.5/(sqrt(2.0)+1)\)
  \Comment{edge length of a regular octagon}
  \State \(c\gets\ b/sqrt(2.0)\)
  \Comment{Cathetus of the outer corner triangle of the octagon}
  \State \(rotAngle\gets\ 45\)
  \Comment{360\textdegree/8}
  \State \(penUp()\)
  \State \(forward(c)\)
  \State \(penDown()\)
  \State \(\Call{forward}{h-c,colorNo}\)
  \State \(penUp()\)
  \State \(backward(h-c)\)
  \State \(penDown()\)
  \State \(right(180)\)
  \State \(\Call{polygonPart}{b,8,true,3,colorNo}\)
  \Comment{Counterclockwise draw 3 edges of an octagoin with edge length b in colour specified by colorNo}
  \State \(left(rotAngle)\)
  \State \(\Call{forward}{h-c,colorNo}\)
  \State \(penUp()\)
  \State \(backward(h)\)
  \State \(penDown()\)
\EndProcedure
\end{algorithmic}
\end{algorithm}


\begin{algorithm}
\caption{qstnMk(2)}
\begin{algorithmic}[5]
\Procedure{qstnMk}{h, colorNo}
\State \Comment{ Draws a question mark in colour specified by colorNo with font height h }
\State \Comment{ from the current turtle position. }
  \Decl{Parameters:}
    \State h: ?
    \State colorNo: ?
  \EndDecl
  \State \(b\gets\ h*0.5/(sqrt(2.0)+1)\)
  \Comment{Octagon edge length}
  \State \(c\gets\ b/sqrt(2.0)\)
  \Comment{Cathetus of the outer corner triangle of the octagon}
  \State \(rotAngle\gets\ 45\)
  \Comment{360\textdegree/8}
  \State \(penUp()\)
  \State \(forward(h-c)\)
  \State \(penDown()\)
  \State \Comment{ Counterclockwise draw 5 edges of an octagon with edge length b }
  \State \Comment{ in the colour endcoded by colorNo }
  \State \(\Call{polygonPart}{b,8,false,5,colorNo}\)
  \State \(\Call{forward}{c,colorNo}\)
  \State \(left(rotAngle)\)
  \State \(\Call{forward}{b/2.0,colorNo}\)
  \State \(penUp()\)
  \State \(forward(c)\)
  \State \(left(90)\)
  \State \(forward(c/2.0)\)
  \State \(penDown()\)
  \State \Comment{ Counterclockwise draw all 4 edges of a squarfe with edge length c }
  \State \Comment{ in the colour endcoded by colorNo }
  \State \(\Call{polygonPart}{c,4,false,4,colorNo}\)
  \State \(penUp()\)
  \State \(forward((c+b)/2.0)\)
  \State \(left(90)\)
  \State \(backward(c)\)
  \State \(penDown()\)
\EndProcedure
\end{algorithmic}
\end{algorithm}


\begin{algorithm}
\caption{digit0(2)}
\begin{algorithmic}[5]
\Procedure{digit0}{h, colorNo}
\State \Comment{ Draws digit 0 in the colour specified by colorNo with font height h }
\State \Comment{ from the current turtle position. }
  \Decl{Parameters:}
    \State h: ?
    \State colorNo: ?
  \EndDecl
  \State \(penUp()\)
  \State \(forward(h/4.0)\)
  \State \(penDown()\)
  \State \(right(45)\)
  \State \(len\gets\ h/sqrt(2)\)
  \State \(\Call{forward}{len,colorNo}\)
  \State \(penUp()\)
  \State \(backward(len)\)
  \State \(left(45)\)
  \State \(backward(h/4.0)\)
  \State \(\Call{letterO}{h,colorNo}\)
\EndProcedure
\end{algorithmic}
\end{algorithm}


\begin{algorithm}
\caption{letterOe(2)}
\begin{algorithmic}[5]
\Procedure{letterOe}{h, colorNo}
\State \Comment{ Draws letter "O in colour specified by colorNo with font height h }
\State \Comment{ from the current turtle position. }
  \Decl{Parameters:}
    \State h: ?
    \State colorNo: ?
  \EndDecl
  \State \(penUp()\)
  \State \(forward(h)\)
  \State \(penDown()\)
  \State \(right(90)\)
  \State \Comment{ Clockwise draw all 4 edges of a square with edge length h/8 }
  \State \Comment{ in the colour endcoded by colorNo }
  \State \(\Call{polygonPart}{h/8,4,false,4,colorNo}\)
  \State \(penUp()\)
  \State \(forward(h/2)\)
  \State \(penDown()\)
  \State \(right(90)\)
  \State \Comment{ Clockwise draw all 4 edges of a square with edge length h/8 }
  \State \Comment{ in the colour endcoded by colorNo }
  \State \(\Call{polygonPart}{h/8,4,false,4,colorNo}\)
  \State \(penUp()\)
  \State \(forward(h)\)
  \State \(penDown()\)
  \State \(right(90)\)
  \State \(penUp()\)
  \State \(forward(h/2)\)
  \State \(penDown()\)
  \State \(right(90)\)
  \State \(\Call{letterO}{h,colorNo}\)
\EndProcedure
\end{algorithmic}
\end{algorithm}


\begin{algorithm}
\caption{letterUe(2)}
\begin{algorithmic}[5]
\Procedure{letterUe}{h, colorNo}
\State \Comment{ Draws letter "U in colour specified by colorNo with font height h }
\State \Comment{ from the current turtle position. }
  \Decl{Parameters:}
    \State h: ?
    \State colorNo: ?
  \EndDecl
  \State \(penUp()\)
  \State \(forward(h)\)
  \State \(right(90)\)
  \State \(forward(max(h/8,1))\)
  \State \(penDown()\)
  \State \Comment{ Clockwise draw all 4 edges of a square with edge length h/16 }
  \State \Comment{ in the colour endcoded by colorNo }
  \State \(\Call{polygonPart}{max(h/16,1),4,false,4,colorNo}\)
  \State \(penUp()\)
  \State \(forward(h/2-2*max(h/8,1)-max(h/16,1))\)
  \State \(penDown()\)
  \State \Comment{ Clockwise draw all 4 edges of a square with edge length h/16 }
  \State \Comment{ in the colour endcoded by colorNo }
  \State \(\Call{polygonPart}{max(h/16,1),4,false,4,colorNo}\)
  \State \(penUp()\)
  \State \(forward(max(h/8,1))\)
  \State \(penDown()\)
  \State \(right(90)\)
  \State \(penUp()\)
  \State \(forward(h)\)
  \State \(right(90)\)
  \State \(forward(h/2)\)
  \State \(penDown()\)
  \State \(right(90)\)
  \State \(\Call{letterU}{h,colorNo}\)
\EndProcedure
\end{algorithmic}
\end{algorithm}


\begin{algorithm}
\caption{drawText(3)}
\begin{algorithmic}[5]
\Procedure{drawText}{text, h, c}
\State \Comment{ Has the turtle draw the given string '{}text\textasciiacute{} with font height '{}h\textasciiacute{} (in }
\State \Comment{ pixels) and the colour coded by integer '{}c\textasciiacute{} from the current Turtle }
\State \Comment{ position to the Turtle canvas. If the turtle looks North then }
\State \Comment{ the text will be written rightwards. In the event, the turtle will be }
\State \Comment{ placed behind the text in original orientation (such that the next text }
\State \Comment{ would be written like a continuation. Colour codes: }
\State \Comment{ 1 = black }
\State \Comment{ 2 = red }
\State \Comment{ 3 = yellow }
\State \Comment{ 4 = green }
\State \Comment{ 5 = cyan }
\State \Comment{ 6 = blue }
\State \Comment{ 7 = pink }
\State \Comment{ 8 = grey }
\State \Comment{ 9 = orange }
\State \Comment{ 10 = violet }
\State \Comment{ All letters (ASCII) will be converted to uppercase, }
\State \Comment{ the set of representable special characters is: decimal digits, }
\State \Comment{ '{}.'{}, '{},'{}, '{}!'{}, '{}?'{}, '{}"A'{}, '{}"O'{}, '{}"U'{}. Other characters will be shown as a small }
\State \Comment{ centred square (dummy character). }
  \Decl{Parameters:}
    \State text: string
    \State h: integer
    \State c: integer
  \EndDecl
  \State \(gap\gets\ h/10.0\)
  \For{\(k \gets 1\) \textbf{to} \(length(text)\) \textbf{by} \(1\)}
    \State \(letter\gets\ uppercase(copy(text,k,1))\)
    \Case{letter}
      \Selector{\)"{}A"{}\(}
        \State \(\Call{letterA}{h,c}\)
      \EndSelector
      \Selector{\)"{}B"{}\(}
        \State \(\Call{letterB}{h,c}\)
      \EndSelector
      \Selector{\)"{}C"{}\(}
        \State \(\Call{letterC}{h,c}\)
      \EndSelector
      \Selector{\)"{}D"{}\(}
        \State \(\Call{letterD}{h,c}\)
      \EndSelector
      \Selector{\)"{}E"{}\(}
        \State \(\Call{letterE}{h,c}\)
      \EndSelector
      \Selector{\)"{}F"{}\(}
        \State \(\Call{letterF}{h,c}\)
      \EndSelector
      \Selector{\)"{}G"{}\(}
        \State \(\Call{letterG}{h,c}\)
      \EndSelector
      \Selector{\)"{}H"{}\(}
        \State \(\Call{letterH}{h,c}\)
      \EndSelector
      \Selector{\)"{}I"{}\(}
        \State \(\Call{letterI}{h,c}\)
      \EndSelector
      \Selector{\)"{}J"{}\(}
        \State \(\Call{letterJ}{h,c}\)
      \EndSelector
      \Selector{\)"{}K"{}\(}
        \State \(\Call{letterK}{h,c}\)
      \EndSelector
      \Selector{\)"{}L"{}\(}
        \State \(\Call{letterL}{h,c}\)
      \EndSelector
      \Selector{\)"{}M"{}\(}
        \State \(\Call{letterM}{h,c}\)
      \EndSelector
      \Selector{\)"{}N"{}\(}
        \State \(\Call{letterN}{h,c}\)
      \EndSelector
      \Selector{\)"{}O"{}\(}
        \State \(\Call{letterO}{h,c}\)
      \EndSelector
      \Selector{\)"{}P"{}\(}
        \State \(\Call{letterP}{h,c}\)
      \EndSelector
      \Selector{\)"{}Q"{}\(}
        \State \(\Call{letterQ}{h,c}\)
      \EndSelector
      \Selector{\)"{}R"{}\(}
        \State \(\Call{letterR}{h,c}\)
      \EndSelector
      \Selector{\)"{}S"{}\(}
        \State \(\Call{letterS}{h,c}\)
      \EndSelector
      \Selector{\)"{}T"{}\(}
        \State \(\Call{letterT}{h,c}\)
      \EndSelector
      \Selector{\)"{}U"{}\(}
        \State \(\Call{letterU}{h,c}\)
      \EndSelector
      \Selector{\)"{}V"{}\(}
        \State \(\Call{letterV}{h,c}\)
      \EndSelector
      \Selector{\)"{}W"{}\(}
        \State \(\Call{letterW}{h,c}\)
      \EndSelector
      \Selector{\)"{}X"{}\(}
        \State \(\Call{letterX}{h,c}\)
      \EndSelector
      \Selector{\)"{}Y"{}\(}
        \State \(\Call{letterY}{h,c}\)
      \EndSelector
      \Selector{\)"{}Z"{}\(}
        \State \(\Call{letterZ}{h,c}\)
      \EndSelector
      \Selector{\)"{}\ "{}\(}
        \State \(\Call{blank}{h,c}\)
      \EndSelector
      \Selector{\)"{}!"{}\(}
        \State \(\Call{exclMk}{h,c}\)
      \EndSelector
      \Selector{\)"{}?"{}\(}
        \State \(\Call{qstnMk}{h,c}\)
      \EndSelector
      \Selector{\)"{}."{}\(}
        \State \(\Call{fullSt}{h,c}\)
      \EndSelector
      \Selector{\)"{},"{}\(}
        \State \(\Call{comma}{h,c}\)
      \EndSelector
      \Selector{\)"{}"A"{}\(}
        \State \(\Call{letterAe}{h,c}\)
      \EndSelector
      \Selector{\)"{}"O"{}\(}
        \State \(\Call{letterOe}{h,c}\)
      \EndSelector
      \Selector{\)"{}"U"{}\(}
        \State \(\Call{letterUe}{h,c}\)
      \EndSelector
      \Selector{\)"{}0"{}\(}
        \State \(\Call{digit0}{h,c}\)
      \EndSelector
      \Selector{\)"{}1"{}\(}
        \State \(\Call{digit1}{h,c}\)
      \EndSelector
      \Selector{\)"{}2"{}\(}
        \State \(\Call{digit2}{h,c}\)
      \EndSelector
      \Selector{\)"{}3"{}\(}
        \State \(\Call{digit3}{h,c}\)
      \EndSelector
      \Selector{\)"{}4"{}\(}
        \State \(\Call{digit4}{h,c}\)
      \EndSelector
      \Selector{\)"{}5"{}\(}
        \State \(\Call{digit5}{h,c}\)
      \EndSelector
      \Selector{\)"{}6"{}\(}
        \State \(\Call{digit6}{h,c}\)
      \EndSelector
      \Selector{\)"{}7"{}\(}
        \State \(\Call{digit7}{h,c}\)
      \EndSelector
      \Selector{\)"{}8"{}\(}
        \State \(\Call{digit8}{h,c}\)
      \EndSelector
      \Selector{\)"{}9"{}\(}
        \State \(\Call{digit9}{h,c}\)
      \EndSelector
      \Other
        \State \(\Call{charDummy}{h,c}\)
      \EndOther
    \EndCase
    \State \(right(90)\)
    \State \(penUp()\)
    \State \(forward(gap)\)
    \State \(penDown()\)
    \State \(left(90)\)
  \EndFor
\EndProcedure
\end{algorithmic}
\end{algorithm}


\State \Comment{ = = = = 8< = = = = = = = = = = = = = = = = = = = = = = = = = = = = = = }


\begin{algorithm}
\caption{TextDemo}
\begin{algorithmic}[5]
\Procedure{TextDemo}{ }
\State \Comment{ Demo program for routine drawText() }
\State \Comment{ Asks the user to enter a text, a wanted text height and colour, }
\State \Comment{ and then draws this string onto the turtle screen. Places every }
\State \Comment{ entered text to a new line. }
  \State \(\)print\((\)"{}This\ is\ a\ demo\ program\ for\ text\ writing\ with\ Turleizer."{}\()\)
  \State \(showTurtle()\)
  \State \(penDown()\)
  \State \(y\gets\ 0\)
  \Repeat
    \State \(\)input\((\)"{}Enter\ some\ text\ (empty\ string\ to\ exit)"{}\(,text)\)
    \State \(text\gets\)"{}"{}\(+text\)
    \Comment{Make sure the content is interpreted as string}
    \If{\(text\neq\)"{}"{}\(\)}
      \Repeat
        \State \(\)input\((\)"{}Height\ of\ the\ text\ (pixels)"{}\(,height)\)
      \Until{\(height\geq\ 5\)}
      \Repeat
        \State \(\)input\((\)"{}Colour\ (1=black,\ 2=red,\ 3=yellow,\ 4=green,\ 5=cyan,\ 6=blue,\ 7=pink,\ 8=gray,\ 9=orange,\ 10=violet)"{}\(,colour)\)
      \Until{\(colour\geq\ 1\wedge\ colour\leq\ 10\)}
      \State \(y\gets\ y+height+2\)
      \State \(gotoXY(0,y-2)\)
      \State \(\Call{drawText}{text,height,colour}\)
    \EndIf
  \Until{\(text=\)"{}"{}\(\)}
  \State \(gotoXY(0,y+15)\)
  \State \(\Call{drawText}{\)"{}Thank\ you,\ bye."{}\(,10,4}\)
  \State \(hideTurtle()\)
\EndProcedure
\end{algorithmic}
\end{algorithm}

\end{document}
