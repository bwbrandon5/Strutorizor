\documentclass[a4paper,10pt]{article}

\usepackage{pseudocode}
\usepackage{ngerman}
\usepackage{amsmath}

\DeclareMathOperator{\oprdiv}{div}
\DeclareMathOperator{\oprshl}{shl}
\DeclareMathOperator{\oprshr}{shr}
\title{Structorizer LaTeX pseudocode Export of ELIZA_2.3.arrz}
\author{Kay Gürtzig}
\date{08.06.2021}

\begin{document}

\begin{pseudocode}{adjustSpelling}{sentence }
\label{adjustSpelling}
\COMMENT{ Cares for correct letter case among others }\\
\PROCEDURE{adjustSpelling}{sentence}
  result\gets\ sentence\\
  position\gets\ 1\\
  \WHILE (position\leq\ length(sentence))\ \AND(copy(sentence,position,1)=\)"{}\ "{}\() \DO
    position\gets\ position+1\\
  \IF position\leq\ length(sentence) \THEN
  \BEGIN
    start\gets\ copy(sentence,1,position)\\
    delete(result,1,position)\\
    insert(uppercase(start),result,1)\\
  \END\\
  \FOREACH word \in \{\)"{}\ i\ "{}\(,\)"{}\ i\textbackslash{}'{}"{}\(\} \DO
  \BEGIN
    position\gets\ pos(word,result)\\
    \WHILE position>0 \DO
    \BEGIN
      delete(result,position+1,1)\\
      insert(\)"{}I"{}\(,result,position+1)\\
      position\gets\ pos(word,result)\\
    \END\\
  \END\\
\ENDPROCEDURE
\end{pseudocode}


\begin{pseudocode}{checkGoodBye}{text, phrases }
\label{checkGoodBye}
\COMMENT{ Checks whether the given text contains some kind of }\\
\COMMENT{ good-bye phrase inducing the end of the conversation }\\
\COMMENT{ and if so writes a correspding good-bye message and }\\
\COMMENT{ returns true, otherwise false }\\
\PROCEDURE{checkGoodBye}{text, phrases}
  \FOREACH pair \in phrases \DO
    \IF pos(pair[0],text)>0 \THEN
    \BEGIN
      \OUTPUT{pair[1]}\\
      \RETURN{true}\\
    \END\\
  \RETURN{false}\\
\ENDPROCEDURE
\end{pseudocode}


\begin{pseudocode}{conjugateStrings}{sentence, key, keyPos, flexions }
\label{conjugateStrings}
\PROCEDURE{conjugateStrings}{sentence, key, keyPos, flexions}
  result\gets\)"{}\ "{}\(+copy(sentence,keyPos+length(key),length(sentence))+\)"{}\ "{}\(\\
  \FOREACH pair \in flexions \DO
  \BEGIN
    left\gets\)"{}"{}\(\\
    right\gets\ result\\
    position\gets\ pos(pair[0],right)\\
    \WHILE position>0 \DO
    \BEGIN
      left\gets\ left+copy(right,1,position-1)+pair[1]\\
      right\gets\ copy(right,position+length(pair[0]),length(right))\\
      position\gets\ pos(pair[0],right)\\
    \END\\
    result\gets\ left+right\\
  \END\\
  \COMMENT{ Eliminate multiple spaces }\\
  position\gets\ pos(\)"{}\ \ "{}\(,result)\\
  \WHILE position>0 \DO
  \BEGIN
    result\gets\ copy(result,1,position-1)+copy(result,position+1,length(result))\\
    position\gets\ pos(\)"{}\ \ "{}\(,result)\\
  \END\\
\ENDPROCEDURE
\end{pseudocode}


\begin{pseudocode}{normalizeInput}{sentence }
\label{normalizeInput}
\COMMENT{ Converts the sentence to lowercase, eliminates all }\\
\COMMENT{ interpunction (i.e. '{},'{}, '{}.'{}, '{};'{}), and pads the }\\
\COMMENT{ sentence among blanks }\\
\PROCEDURE{normalizeInput}{sentence}
  sentence\gets\ lowercase(sentence)\\
  \FOREACH symbol \in \{\)'{}.'{}\(,\)'{},'{}\(,\)'{};'{}\(,\)'{}!'{}\(,\)'{}?'{}\(\} \DO
  \BEGIN
    position\gets\ pos(symbol,sentence)\\
    \WHILE position>0 \DO
    \BEGIN
      sentence\gets\ copy(sentence,1,position-1)+copy(sentence,position+1,length(sentence))\\
      position\gets\ pos(symbol,sentence)\\
    \END\\
  \END\\
  result\gets\)"{}\ "{}\(+sentence+\)"{}\ "{}\(\\
\ENDPROCEDURE
\end{pseudocode}


\begin{pseudocode}{setupGoodByePhrases}{ }
\label{setupGoodByePhrases}
\PROCEDURE{setupGoodByePhrases}{}
  phrases[0]\gets\{\)"{}\ shut"{}\(,\)"{}Okay.\ If\ you\ feel\ that\ way\ I\textbackslash{}'{}ll\ shut\ up.\ ...\ Your\ choice."{}\(\}\\
  phrases[1]\gets\{\)"{}bye"{}\(,\)"{}Well,\ let\textbackslash{}'{}s\ end\ our\ talk\ for\ now.\ See\ you\ later.\ Bye."{}\(\}\\
  \RETURN{phrases}\\
\ENDPROCEDURE
\end{pseudocode}


\begin{pseudocode}{setupReflexions}{ }
\label{setupReflexions}
\COMMENT{ Returns an array of pairs of mutualy substitutable  }\\
\PROCEDURE{setupReflexions}{}
  reflexions[0]\gets\{\)"{}\ are\ "{}\(,\)"{}\ am\ "{}\(\}\\
  reflexions[1]\gets\{\)"{}\ were\ "{}\(,\)"{}\ was\ "{}\(\}\\
  reflexions[2]\gets\{\)"{}\ you\ "{}\(,\)"{}\ I\ "{}\(\}\\
  reflexions[3]\gets\{\)"{}\ your"{}\(,\)"{}\ my"{}\(\}\\
  reflexions[4]\gets\{\)"{}\ i\textbackslash{}'{}ve\ "{}\(,\)"{}\ you\textbackslash{}'{}ve\ "{}\(\}\\
  reflexions[5]\gets\{\)"{}\ i\textbackslash{}'{}m\ "{}\(,\)"{}\ you\textbackslash{}'{}re\ "{}\(\}\\
  reflexions[6]\gets\{\)"{}\ me\ "{}\(,\)"{}\ you\ "{}\(\}\\
  reflexions[7]\gets\{\)"{}\ my\ "{}\(,\)"{}\ your\ "{}\(\}\\
  reflexions[8]\gets\{\)"{}\ i\ "{}\(,\)"{}\ you\ "{}\(\}\\
  reflexions[9]\gets\{\)"{}\ am\ "{}\(,\)"{}\ are\ "{}\(\}\\
  \RETURN{reflexions}\\
\ENDPROCEDURE
\end{pseudocode}


\begin{pseudocode}{setupReplies}{ }
\label{setupReplies}
\COMMENT{ This routine sets up the reply rings addressed by the key words defined in }\\
\COMMENT{ routine `setupKeywords()´ and mapped hitherto by the cross table defined }\\
\COMMENT{ in `setupMapping()´ }\\
\PROCEDURE{setupReplies}{}
  var\ replies:array\ of\ array\ of\ String\\
  \COMMENT{ We start with the highest index for performance reasons }\\
  \COMMENT{ (is to avoid frequent array resizing) }\\
  replies[29]\gets\{\)"{}Say,\ do\ you\ have\ any\ psychological\ problems?"{}\(,\)"{}What\ does\ that\ suggest\ to\ you?"{}\(,\)"{}I\ see."{}\(,\)"{}I'{}m\ not\ sure\ I\ understand\ you\ fully."{}\(,\)"{}Come\ come\ elucidate\ your\ thoughts."{}\(,\)"{}Can\ you\ elaborate\ on\ that?"{}\(,\)"{}That\ is\ quite\ interesting."{}\(\}\\
  replies[0]\gets\{\)"{}Don'{}t\ you\ believe\ that\ I\ can*?"{}\(,\)"{}Perhaps\ you\ would\ like\ to\ be\ like\ me?"{}\(,\)"{}You\ want\ me\ to\ be\ able\ to*?"{}\(\}\\
  replies[1]\gets\{\)"{}Perhaps\ you\ don'{}t\ want\ to*?"{}\(,\)"{}Do\ you\ want\ to\ be\ able\ to*?"{}\(\}\\
  replies[2]\gets\{\)"{}What\ makes\ you\ think\ I\ am*?"{}\(,\)"{}Does\ it\ please\ you\ to\ believe\ I\ am*?"{}\(,\)"{}Perhaps\ you\ would\ like\ to\ be*?"{}\(,\)"{}Do\ you\ sometimes\ wish\ you\ were*?"{}\(\}\\
  replies[3]\gets\{\)"{}Don'{}t\ you\ really*?"{}\(,\)"{}Why\ don'{}t\ you*?"{}\(,\)"{}Do\ you\ wish\ to\ be\ able\ to*?"{}\(,\)"{}Does\ that\ trouble\ you*?"{}\(\}\\
  replies[4]\gets\{\)"{}Do\ you\ often\ feel*?"{}\(,\)"{}Are\ you\ afraid\ of\ feeling*?"{}\(,\)"{}Do\ you\ enjoy\ feeling*?"{}\(\}\\
  replies[5]\gets\{\)"{}Do\ you\ really\ believe\ I\ don'{}t*?"{}\(,\)"{}Perhaps\ in\ good\ time\ I\ will*."{}\(,\)"{}Do\ you\ want\ me\ to*?"{}\(\}\\
  replies[6]\gets\{\)"{}Do\ you\ think\ you\ should\ be\ able\ to*?"{}\(,\)"{}Why\ can'{}t\ you*?"{}\(\}\\
  replies[7]\gets\{\)"{}Why\ are\ you\ interested\ in\ whether\ or\ not\ I\ am*?"{}\(,\)"{}Would\ you\ prefer\ if\ I\ were\ not*?"{}\(,\)"{}Perhaps\ in\ your\ fantasies\ I\ am*?"{}\(\}\\
  replies[8]\gets\{\)"{}How\ do\ you\ know\ you\ can'{}t*?"{}\(,\)"{}Have\ you\ tried?"{}\(,\)"{}Perhaps\ you\ can\ now*."{}\(\}\\
  replies[9]\gets\{\)"{}Did\ you\ come\ to\ me\ because\ you\ are*?"{}\(,\)"{}How\ long\ have\ you\ been*?"{}\(,\)"{}Do\ you\ believe\ it\ is\ normal\ to\ be*?"{}\(,\)"{}Do\ you\ enjoy\ being*?"{}\(\}\\
  replies[10]\gets\{\)"{}We\ were\ discussing\ you--not\ me."{}\(,\)"{}Oh,\ I*."{}\(,\)"{}You'{}re\ not\ really\ talking\ about\ me,\ are\ you?"{}\(\}\\
  replies[11]\gets\{\)"{}What\ would\ it\ mean\ to\ you\ if\ you\ got*?"{}\(,\)"{}Why\ do\ you\ want*?"{}\(,\)"{}Suppose\ you\ soon\ got*..."{}\(,\)"{}What\ if\ you\ never\ got*?"{}\(,\)"{}I\ sometimes\ also\ want*."{}\(\}\\
  replies[12]\gets\{\)"{}Why\ do\ you\ ask?"{}\(,\)"{}Does\ that\ question\ interest\ you?"{}\(,\)"{}What\ answer\ would\ please\ you\ the\ most?"{}\(,\)"{}What\ do\ you\ think?"{}\(,\)"{}Are\ such\ questions\ on\ your\ mind\ often?"{}\(,\)"{}What\ is\ it\ that\ you\ really\ want\ to\ know?"{}\(,\)"{}Have\ you\ asked\ anyone\ else?"{}\(,\)"{}Have\ you\ asked\ such\ questions\ before?"{}\(,\)"{}What\ else\ comes\ to\ mind\ when\ you\ ask\ that?"{}\(\}\\
  replies[13]\gets\{\)"{}Names\ don'{}t\ interest\ me."{}\(,\)"{}I\ don'{}t\ care\ about\ names\ --\ please\ go\ on."{}\(\}\\
  replies[14]\gets\{\)"{}Is\ that\ the\ real\ reason?"{}\(,\)"{}Don'{}t\ any\ other\ reasons\ come\ to\ mind?"{}\(,\)"{}Does\ that\ reason\ explain\ anything\ else?"{}\(,\)"{}What\ other\ reasons\ might\ there\ be?"{}\(\}\\
  replies[15]\gets\{\)"{}Please\ don'{}t\ apologize!"{}\(,\)"{}Apologies\ are\ not\ necessary."{}\(,\)"{}What\ feelings\ do\ you\ have\ when\ you\ apologize?"{}\(,\)"{}Don'{}t\ be\ so\ defensive!"{}\(\}\\
  replies[16]\gets\{\)"{}What\ does\ that\ dream\ suggest\ to\ you?"{}\(,\)"{}Do\ you\ dream\ often?"{}\(,\)"{}What\ persons\ appear\ in\ your\ dreams?"{}\(,\)"{}Are\ you\ disturbed\ by\ your\ dreams?"{}\(\}\\
  replies[17]\gets\{\)"{}How\ do\ you\ do\ ...please\ state\ your\ problem."{}\(\}\\
  replies[18]\gets\{\)"{}You\ don'{}t\ seem\ quite\ certain."{}\(,\)"{}Why\ the\ uncertain\ tone?"{}\(,\)"{}Can'{}t\ you\ be\ more\ positive?"{}\(,\)"{}You\ aren'{}t\ sure?"{}\(,\)"{}Don'{}t\ you\ know?"{}\(\}\\
  replies[19]\gets\{\)"{}Are\ you\ saying\ no\ just\ to\ be\ negative?"{}\(,\)"{}You\ are\ being\ a\ bit\ negative."{}\(,\)"{}Why\ not?"{}\(,\)"{}Are\ you\ sure?"{}\(,\)"{}Why\ no?"{}\(\}\\
  replies[20]\gets\{\)"{}Why\ are\ you\ concerned\ about\ my*?"{}\(,\)"{}What\ about\ your\ own*?"{}\(\}\\
  replies[21]\gets\{\)"{}Can\ you\ think\ of\ a\ specific\ example?"{}\(,\)"{}When?"{}\(,\)"{}What\ are\ you\ thinking\ of?"{}\(,\)"{}Really,\ always?"{}\(\}\\
  replies[22]\gets\{\)"{}Do\ you\ really\ think\ so?"{}\(,\)"{}But\ you\ are\ not\ sure\ you*?"{}\(,\)"{}Do\ you\ doubt\ you*?"{}\(\}\\
  replies[23]\gets\{\)"{}In\ what\ way?"{}\(,\)"{}What\ resemblance\ do\ you\ see?"{}\(,\)"{}What\ does\ the\ similarity\ suggest\ to\ you?"{}\(,\)"{}What\ other\ connections\ do\ you\ see?"{}\(,\)"{}Could\ there\ really\ be\ some\ connection?"{}\(,\)"{}How?"{}\(,\)"{}You\ seem\ quite\ positive."{}\(\}\\
  replies[24]\gets\{\)"{}Are\ you\ sure?"{}\(,\)"{}I\ see."{}\(,\)"{}I\ understand."{}\(\}\\
  replies[25]\gets\{\)"{}Why\ do\ you\ bring\ up\ the\ topic\ of\ friends?"{}\(,\)"{}Do\ your\ friends\ worry\ you?"{}\(,\)"{}Do\ your\ friends\ pick\ on\ you?"{}\(,\)"{}Are\ you\ sure\ you\ have\ any\ friends?"{}\(,\)"{}Do\ you\ impose\ on\ your\ friends?"{}\(,\)"{}Perhaps\ your\ love\ for\ friends\ worries\ you."{}\(\}\\
  replies[26]\gets\{\)"{}Do\ computers\ worry\ you?"{}\(,\)"{}Are\ you\ talking\ about\ me\ in\ particular?"{}\(,\)"{}Are\ you\ frightened\ by\ machines?"{}\(,\)"{}Why\ do\ you\ mention\ computers?"{}\(,\)"{}What\ do\ you\ think\ machines\ have\ to\ do\ with\ your\ problem?"{}\(,\)"{}Don'{}t\ you\ think\ computers\ can\ help\ people?"{}\(,\)"{}What\ is\ it\ about\ machines\ that\ worries\ you?"{}\(\}\\
  replies[27]\gets\{\)"{}Do\ you\ sometimes\ feel\ uneasy\ without\ a\ smartphone?"{}\(,\)"{}Have\ you\ had\ these\ phantasies\ before?"{}\(,\)"{}Does\ the\ world\ seem\ more\ real\ for\ you\ via\ apps?"{}\(\}\\
  replies[28]\gets\{\)"{}Tell\ me\ more\ about\ your\ family."{}\(,\)"{}Who\ else\ in\ your\ family*?"{}\(,\)"{}What\ does\ family\ relations\ mean\ for\ you?"{}\(,\)"{}Come\ on,\ How\ old\ are\ you?"{}\(\}\\
  setupReplies\gets\ replies\\
\ENDPROCEDURE
\end{pseudocode}


\begin{pseudocode}{checkRepetition}{history, newInput }
\label{checkRepetition}
\COMMENT{ Checks whether newInput has occurred among the recently cached }\\
\COMMENT{ input strings in the histArray component of history and updates the history. }\\
\PROCEDURE{checkRepetition}{history, newInput}
  hasOccurred\gets\FALSE\\
  \IF length(newInput)>4 \THEN
  \BEGIN
    histDepth\gets\ length(history.histArray)\\
    \FOR i \gets 0 \TO histDepth-1  \DO
      \IF newInput=history.histArray[i] \THEN
        hasOccurred\gets\TRUE\\
    history.histArray[history.histIndex]\gets\ newInput\\
    history.histIndex\gets(history.histIndex+1)\bmod(histDepth)\\
  \END\\
  \RETURN{hasOccurred}\\
\ENDPROCEDURE
\end{pseudocode}


\begin{pseudocode}{findKeyword}{keyMap, sentence }
\label{findKeyword}
\COMMENT{ Looks for the occurrence of the first of the strings }\\
\COMMENT{ contained in keywords within the given sentence (in }\\
\COMMENT{ array order). }\\
\COMMENT{ Returns an array of }\\
\COMMENT{ 0: the index of the first identified keyword (if any, otherwise -1), }\\
\COMMENT{ 1: the position inside sentence (0 if not found) }\\
\PROCEDURE{findKeyword}{keyMap, sentence}
  \COMMENT{ Contains the index of the keyword and its position in sentence }\\
  result\gets\{-1,0\}\\
  i\gets\ 0\\
  \WHILE (result[0]<0)\ \AND(i<length(keyMap)) \DO
  \BEGIN
    var\ entry:KeyMapEntry\gets\ keyMap[i]\\
    position\gets\ pos(entry.keyword,sentence)\\
    \IF position>0 \THEN
    \BEGIN
      result[0]\gets\ i\\
      result[1]\gets\ position\\
    \END\\
    i\gets\ i+1\\
  \END\\
\ENDPROCEDURE
\end{pseudocode}


\begin{pseudocode}{setupKeywords}{ }
\label{setupKeywords}
\COMMENT{ The lower the index the higher the rank of the keyword (search is sequential). }\\
\COMMENT{ The index of the first keyword found in a user sentence maps to a respective }\\
\COMMENT{ reply ring as defined in `setupReplies()´. }\\
\PROCEDURE{setupKeywords}{}
  \COMMENT{ The empty key string (last entry) is the default clause - will always be found }\\
  keywords[39]\gets\ KeyMapEntry\{\)"{}"{}\(,29\}\\
  keywords[0]\gets\ KeyMapEntry\{\)"{}can\ you\ "{}\(,0\}\\
  keywords[1]\gets\ KeyMapEntry\{\)"{}can\ i\ "{}\(,1\}\\
  keywords[2]\gets\ KeyMapEntry\{\)"{}you\ are\ "{}\(,2\}\\
  keywords[3]\gets\ KeyMapEntry\{\)"{}you\textbackslash{}'{}re\ "{}\(,2\}\\
  keywords[4]\gets\ KeyMapEntry\{\)"{}i\ don'{}t\ "{}\(,3\}\\
  keywords[5]\gets\ KeyMapEntry\{\)"{}i\ feel\ "{}\(,4\}\\
  keywords[6]\gets\ KeyMapEntry\{\)"{}why\ don\textbackslash{}'{}t\ you\ "{}\(,5\}\\
  keywords[7]\gets\ KeyMapEntry\{\)"{}why\ can\textbackslash{}'{}t\ i\ "{}\(,6\}\\
  keywords[8]\gets\ KeyMapEntry\{\)"{}are\ you\ "{}\(,7\}\\
  keywords[9]\gets\ KeyMapEntry\{\)"{}i\ can\textbackslash{}'{}t\ "{}\(,8\}\\
  keywords[10]\gets\ KeyMapEntry\{\)"{}i\ am\ "{}\(,9\}\\
  keywords[11]\gets\ KeyMapEntry\{\)"{}i\textbackslash{}'{}m\ "{}\(,9\}\\
  keywords[12]\gets\ KeyMapEntry\{\)"{}you\ "{}\(,10\}\\
  keywords[13]\gets\ KeyMapEntry\{\)"{}i\ want\ "{}\(,11\}\\
  keywords[14]\gets\ KeyMapEntry\{\)"{}what\ "{}\(,12\}\\
  keywords[15]\gets\ KeyMapEntry\{\)"{}how\ "{}\(,12\}\\
  keywords[16]\gets\ KeyMapEntry\{\)"{}who\ "{}\(,12\}\\
  keywords[17]\gets\ KeyMapEntry\{\)"{}where\ "{}\(,12\}\\
  keywords[18]\gets\ KeyMapEntry\{\)"{}when\ "{}\(,12\}\\
  keywords[19]\gets\ KeyMapEntry\{\)"{}why\ "{}\(,12\}\\
  keywords[20]\gets\ KeyMapEntry\{\)"{}name\ "{}\(,13\}\\
  keywords[21]\gets\ KeyMapEntry\{\)"{}cause\ "{}\(,14\}\\
  keywords[22]\gets\ KeyMapEntry\{\)"{}sorry\ "{}\(,15\}\\
  keywords[23]\gets\ KeyMapEntry\{\)"{}dream\ "{}\(,16\}\\
  keywords[24]\gets\ KeyMapEntry\{\)"{}hello\ "{}\(,17\}\\
  keywords[25]\gets\ KeyMapEntry\{\)"{}hi\ "{}\(,17\}\\
  keywords[26]\gets\ KeyMapEntry\{\)"{}maybe\ "{}\(,18\}\\
  keywords[27]\gets\ KeyMapEntry\{\)"{}\ no"{}\(,19\}\\
  keywords[28]\gets\ KeyMapEntry\{\)"{}your\ "{}\(,20\}\\
  keywords[29]\gets\ KeyMapEntry\{\)"{}always\ "{}\(,21\}\\
  keywords[30]\gets\ KeyMapEntry\{\)"{}think\ "{}\(,22\}\\
  keywords[31]\gets\ KeyMapEntry\{\)"{}alike\ "{}\(,23\}\\
  keywords[32]\gets\ KeyMapEntry\{\)"{}yes\ "{}\(,24\}\\
  keywords[33]\gets\ KeyMapEntry\{\)"{}friend\ "{}\(,25\}\\
  keywords[34]\gets\ KeyMapEntry\{\)"{}computer"{}\(,26\}\\
  keywords[35]\gets\ KeyMapEntry\{\)"{}bot\ "{}\(,26\}\\
  keywords[36]\gets\ KeyMapEntry\{\)"{}smartphone"{}\(,27\}\\
  keywords[37]\gets\ KeyMapEntry\{\)"{}father\ "{}\(,28\}\\
  keywords[38]\gets\ KeyMapEntry\{\)"{}mother\ "{}\(,28\}\\
  \RETURN{keywords}\\
\ENDPROCEDURE
\end{pseudocode}


\begin{pseudocode}{ELIZA}{ }
\label{ELIZA}
\COMMENT{ Concept and lisp implementation published by Joseph Weizenbaum (MIT): }\\
\COMMENT{ "{}ELIZA - A Computer Program For the Study of Natural Language Communication Between Man and Machine"{} - In: }\\
\COMMENT{ Computational Linguistis 1(1966)9, pp. 36-45 }\\
\COMMENT{ Revision history: }\\
\COMMENT{ 2016-10-06 Initial version }\\
\COMMENT{ 2017-03-29 Two diagrams updated (comments translated to English) }\\
\COMMENT{ 2017-03-29 More keywords and replies added }\\
\COMMENT{ 2019-03-14 Replies and mapping reorganised for easier maintenance }\\
\COMMENT{ 2019-03-15 key map joined from keyword array and index map }\\
\COMMENT{ 2019-03-28 Keyword "{}bot"{} inserted (same reply ring as "{}computer"{}) }\\
\COMMENT{ 2019-11-28 New global type "{}History"{} (to ensure a homogenous array) }\\
\MAIN
  \COMMENT{ Title information }\\
  \OUTPUT{\)"{}*************\ ELIZA\ **************"{}\(}\\
  \OUTPUT{\)"{}*\ Original\ design\ by\ J.\ Weizenbaum"{}\(}\\
  \OUTPUT{\)"{}**********************************"{}\(}\\
  \OUTPUT{\)"{}*\ Adapted\ for\ Basic\ on\ IBM\ PC\ by"{}\(}\\
  \OUTPUT{\)"{}*\ -\ Patricia\ Danielson"{}\(}\\
  \OUTPUT{\)"{}*\ -\ Paul\ Hashfield"{}\(}\\
  \OUTPUT{\)"{}**********************************"{}\(}\\
  \OUTPUT{\)"{}*\ Adapted\ for\ Structorizer\ by"{}\(}\\
  \OUTPUT{\)"{}*\ -\ Kay\ G"urtzig\ /\ FH\ Erfurt\ 2016"{}\(}\\
  \OUTPUT{\)"{}*\ Version:\ 2.3\ (2020-02-24)"{}\(}\\
  \OUTPUT{\)"{}*\ (Requires\ at\ least\ Structorizer\ 3.30-03\ to\ run)"{}\(}\\
  \OUTPUT{\)"{}**********************************"{}\(}\\
  \COMMENT{ Stores the last five inputs of the user in a ring buffer, }\\
  \COMMENT{ the second component is the rolling (over-)write index. }\\
  history\gets\ History\{\{\)"{}"{}\(,\)"{}"{}\(,\)"{}"{}\(,\)"{}"{}\(,\)"{}"{}\(\},0\}\\
  const\ replies\gets\CALL{setupReplies}{}\\
  const\ reflexions\gets\CALL{setupReflexions}{}\\
  const\ byePhrases\gets\CALL{setupGoodByePhrases}{}\\
  const\ keyMap\gets\CALL{setupKeywords}{}\\
  offsets[length(keyMap)-1]\gets\ 0\\
  isGone\gets\FALSE\\
  \COMMENT{ Starter }\\
  \OUTPUT{\)"{}Hi!\ I\textbackslash{}'{}m\ your\ new\ therapist.\ My\ name\ is\ Eliza.\ What\textbackslash{}'{}s\ your\ problem?"{}\(}\\
  \REPEAT
    \)input\(userInput\\
    \COMMENT{ Converts the input to lowercase, cuts out interpunctation }\\
    \COMMENT{ and pads the string }\\
    userInput\gets\CALL{normalizeInput}{userInput}\\
    isGone\gets\CALL{checkGoodBye}{userInput,byePhrases}\\
    \IF \NOT\ isGone \THEN
    \BEGIN
      reply\gets\)"{}Please\ don\textbackslash{}'{}t\ repeat\ yourself!"{}\(\\
      isRepeated\gets\CALL{checkRepetition}{history,userInput}\\
      \IF \NOT\ isRepeated \THEN
      \BEGIN
        findInfo\gets\CALL{findKeyword}{keyMap,userInput}\\
        keyIndex\gets\ findInfo[0]\\
        \IF keyIndex<0 \THEN
        \BEGIN
          \COMMENT{ Should never happen... }\\
          keyIndex\gets\ length(keyMap)-1\\
        \END\\
        var\ entry:KeyMapEntry\gets\ keyMap[keyIndex]\\
        \COMMENT{ Variable part of the reply }\\
        varPart\gets\)"{}"{}\(\\
        \IF length(entry.keyword)>0 \THEN
          varPart\gets\CALL{conjugateStrings}{userInput,entry.keyword,findInfo[1],reflexions}\\
        replyRing\gets\ replies[entry.index]\\
        reply\gets\ replyRing[offsets[keyIndex]]\\
        offsets[keyIndex]\gets(offsets[keyIndex]+1)\bmod\ length(replyRing)\\
        posAster\gets\ pos(\)"{}*"{}\(,reply)\\
        \IF posAster>0 \THEN
          \IF varPart=\)"{}\ "{}\( \THEN
            reply\gets\)"{}You\ will\ have\ to\ elaborate\ more\ for\ me\ to\ help\ you."{}\(\\
          \ELSE
          \BEGIN
            delete(reply,posAster,1)\\
            insert(varPart,reply,posAster)\\
          \END\\
        reply\gets\CALL{adjustSpelling}{reply}\\
      \END\\
      \OUTPUT{reply}\\
    \END\\
  \UNTIL isGone\\
\ENDMAIN
\end{pseudocode}


\begin{pseudocode}{History}{ }
\label{History}
\COMMENT{ Defines a history type, consisting of an array and a rotating index }\\

  \COMMENT{ histArray contains the most recent user replies as ring buffer; }\\
  \COMMENT{ histIndex is the index where the next reply is to be stored (= index of the oldest }\\
  \COMMENT{ cached user reply). }\\
  \COMMENT{ Note: The depth of the history is to be specified by initializing a variable of this type, }\\
  \COMMENT{ e.g. for a history of depth 5: }\\
  \COMMENT{ myhistory \textless- History\{\{"{}"{}, "{}"{}, "{}"{}, "{}"{}, "{}"{}\}, 0\} }\\
  type\ History=record\{histArray:array\ of\ string;histIndex:int\}\\

\end{pseudocode}


\begin{pseudocode}{KeyMapEntry}{ }
\label{KeyMapEntry}
\COMMENT{ Defines the map entry type of the same name }\\

  \COMMENT{ Associates a key word in the text with an index in the reply ring array }\\
  type\ KeyMapEntry=record\{keyword:string;index:int\}\\

\end{pseudocode}

\end{document}
