\documentclass[a4paper,10pt]{article}

\usepackage{pseudocode}
\usepackage{ngerman}
\usepackage{amsmath}

\DeclareMathOperator{\oprdiv}{div}
\DeclareMathOperator{\oprshl}{shl}
\DeclareMathOperator{\oprshr}{shr}
\title{Structorizer LaTeX pseudocode Export of FileApiGroupTest.arrz}
\author{Kay G"urtzig}
\date{09.06.2021}

\begin{document}

\begin{pseudocode}{ComputeSum}{ }
\label{ComputeSum}
\COMMENT{ Computes the sum and average of the numbers read from a user-specified }\\
\COMMENT{ text file (which might have been created via generateRandomNumberFile(4)). }\\
\COMMENT{  }\\
\COMMENT{ This program is part of an arrangement used to test group code export (issue }\\
\COMMENT{ \#828) with FileAPI dependency. }\\
\COMMENT{ The input check loop has been disabled (replaced by a simple unchecked input }\\
\COMMENT{ instruction) in order to test the effect of indirect FileAPI dependency (only the }\\
\COMMENT{ called subroutine directly requires FileAPI now). }\\
\MAIN
  fileNo\gets\ 1000;\\
  \COMMENT{ Disable this if you enable the loop below! }\\
  \)input\(\)"{}Name/path\ of\ the\ number\ file"{}\(file\_name;\\
  \IF fileNo>0 \THEN
  \BEGIN
    values\gets\{\};\\
    nValues\gets\ 0;\\
    \textbf{try} \BEGIN
      nValues\gets\CALL{readNumbers}{file\_name,values,1000};\\
    \END\\
    \textbf{catch}\ (failure)\BEGIN
      \OUTPUT{failure};\\
      \EXIT -7;\\
    \END\\
    sum\gets\ 0.0;\\
    \FOR k \gets 0 \TO nValues-1  \DO
      sum\gets\ sum+values[k];\\
    \OUTPUT{\)"{}sum\ =\ "{}\(,sum};\\
    \OUTPUT{\)"{}average\ =\ "{}\(,sum/nValues};\\
  \END\\
\ENDMAIN
\end{pseudocode}


\begin{pseudocode}{DrawRandomHistogram}{ }
\label{DrawRandomHistogram}
\COMMENT{ Reads a random number file and draws a histogram accotrding to the }\\
\COMMENT{ user specifications }\\
\MAIN
  fileNo\gets-10;\\
  \REPEAT
    \)input\(\)"{}Name/path\ of\ the\ number\ file"{}\(file\_name;\\
    fileNo\gets\ fileOpen(file\_name);\\
  \UNTIL fileNo>0\ \OR\ file\_name=\)"{}"{}\(;\\
  \IF fileNo>0 \THEN
  \BEGIN
    fileClose(fileNo);\\
    \)input\(\)"{}number\ of\ intervals"{}\(,nIntervals;\\
    \COMMENT{ Initialize the interval counters }\\
    \FOR k \gets 0 \TO nIntervals-1  \DO
      count[k]\gets\ 0;\\
    \COMMENT{ Index of the most populated interval }\\
    kMaxCount\gets\ 0;\\
    numberArray\gets\{\};\\
    nObtained\gets\ 0;\\
    \textbf{try} \BEGIN
      nObtained\gets\CALL{readNumbers}{file\_name,numberArray,10000};\\
    \END\\
    \textbf{catch}\ (failure)\BEGIN
      \OUTPUT{failure};\\
    \END\\
    \IF nObtained>0 \THEN
    \BEGIN
      min\gets\ numberArray[0];\\
      max\gets\ numberArray[0];\\
      \FOR i \gets 1 \TO nObtained-1  \DO
        \IF numberArray[i]<min \THEN
          min\gets\ numberArray[i];\\
        \ELSE
          \IF numberArray[i]>max \THEN
            max\gets\ numberArray[i];\\
      \COMMENT{ Interval width }\\
      width\gets(max-min)*1.0/nIntervals;\\
      \FOR i \gets 0 \TO nObtained-1  \DO
      \BEGIN
        value\gets\ numberArray[i];\\
        k\gets\ 1;\\
        \WHILE k<nIntervals\ \AND\ value>min+k*width \DO
          k\gets\ k+1;\\
        count[k-1]\gets\ count[k-1]+1;\\
        \IF count[k-1]>count[kMaxCount] \THEN
          kMaxCount\gets\ k-1;\\
      \END\\
      \CALL{drawBarChart}{count,nIntervals};\\
      \OUTPUT{\)"{}Interval\ with\ max\ count:\ "{}\(,kMaxCount,\)"{}\ ("{}\(,count[kMaxCount],\)"{})"{}\(};\\
      \FOR k \gets 0 \TO nIntervals-1  \DO
        \OUTPUT{count[k],\)"{}\ numbers\ in\ interval\ "{}\(,k,\)"{}\ ("{}\(,min+k*width,\)"{}\ ...\ "{}\(,min+(k+1)*width,\)"{})"{}\(};\\
    \END\\
    \ELSE
      \OUTPUT{\)"{}No\ numbers\ read."{}\(};\\
  \END\\
\ENDMAIN
\end{pseudocode}


\begin{pseudocode}{drawBarChart}{values, nValues }
\label{drawBarChart}
\COMMENT{ Draws a bar chart from the array "{}values"{} of size nValues. }\\
\COMMENT{ Turtleizer must be activated and will scale the chart into a square of }\\
\COMMENT{ 500 x 500 pixels }\\
\COMMENT{ Note: The function is not robust against empty array or totally equal values. }\\
\PROCEDURE{drawBarChart}{values, nValues}
  \COMMENT{ Used range of the Turtleizer screen }\\
  const\ xSize\gets\ 500;\\
  const\ ySize\gets\ 500;\\
  kMin\gets\ 0;\\
  kMax\gets\ 0;\\
  \FOR k \gets 1 \TO nValues-1  \DO
    \IF values[k]>values[kMax] \THEN
      kMax\gets\ k;\\
    \ELSE
      \IF values[k]<values[kMin] \THEN
        kMin\gets\ k;\\
  valMin\gets\ values[kMin];\\
  valMax\gets\ values[kMax];\\
  yScale\gets\ valMax*1.0/(ySize-1);\\
  yAxis\gets\ ySize-1;\\
  \IF valMin<0 \THEN
    \IF valMax>0 \THEN
    \BEGIN
      yAxis\gets\ valMax*ySize*1.0/(valMax-valMin);\\
      yScale\gets(valMax-valMin)*1.0/(ySize-1);\\
    \END\\
    \ELSE
    \BEGIN
      yAxis\gets\ 1;\\
      yScale\gets\ valMin*1.0/(ySize-1);\\
    \END\\
  \COMMENT{ draw coordinate axes }\\
  gotoXY(1,ySize-1);\\
  forward(ySize-1);\\
  penUp();\\
  backward(yAxis);\\
  right(90);\\
  penDown();\\
  forward(xSize-1);\\
  penUp();\\
  backward(xSize-1);\\
  stripeWidth\gets\ xSize/nValues;\\
  \FOR k \gets 0 \TO nValues-1  \DO
  \BEGIN
    stripeHeight\gets\ values[k]*1.0/yScale;\\
    discr607e3da0 <- k mod 3;\\
    \IF discr607e3da0=0 \THEN
      setPenColor(255,0,0);\\
    \ELSEIF discr607e3da0=1 \THEN
      setPenColor(0,255,0);\\
    \ELSEIF discr607e3da0=2 \THEN
      setPenColor(0,0,255);\\
    fd(1);\\
    left(90);\\
    penDown();\\
    fd(stripeHeight);\\
    right(90);\\
    fd(stripeWidth-1);\\
    right(90);\\
    forward(stripeHeight);\\
    left(90);\\
    penUp();\\
  \END\\
\ENDPROCEDURE
\end{pseudocode}


\begin{pseudocode}{generateRandomNumberFile}{fileName, count, minVal, maxVal }
\label{generateRandomNumberFile}
\COMMENT{ Writes a text file with name fileName, consisting of count lines, each containing }\\
\COMMENT{ a random number in the range from minVal to maxVal (both inclusive). }\\
\COMMENT{ Returns either the number of written number if all went well or some potential }\\
\COMMENT{ error code if something went wrong. }\\
\PROCEDURE{generateRandomNumberFile}{fileName, count, minVal, maxVal}
  fileNo\gets\ fileCreate(fileName);\\
  \IF fileNo\leq\ 0 \THEN
    \RETURN{fileNo};\\
  \textbf{try} \BEGIN
    \FOR k \gets 1 \TO count  \DO
    \BEGIN
      number\gets\ minVal+random(maxVal-minVal+1);\\
      fileWriteLine(fileNo,number);\\
    \END\\
  \END\\
  \textbf{catch}\ (error)\BEGIN
    \OUTPUT{error};\\
    \RETURN{-7};\\
  \END\\
  \textbf{finally} \BEGIN
    fileClose(fileNo);\\
  \END\\
  \RETURN{count};\\
\ENDPROCEDURE
\end{pseudocode}


\begin{pseudocode}{readNumbers}{fileName, numbers, maxNumbers }
\label{readNumbers}
\COMMENT{ Tries to read as many integer values as possible upto maxNumbers }\\
\COMMENT{ from file fileName into the given array numbers. }\\
\COMMENT{ Returns the number of the actually read numbers. May cause an exception. }\\
\PROCEDURE{readNumbers}{fileName, numbers, maxNumbers}
  nNumbers\gets\ 0;\\
  fileNo\gets\ fileOpen(fileName);\\
  \IF fileNo\leq\ 0 \THEN
    \textbf{throw}\ \)"{}File could not be opened!"{}\(;\\
  \textbf{try} \BEGIN
    \WHILE \NOT\ fileEOF(fileNo)\ \AND\ nNumbers<maxNumbers \DO
    \BEGIN
      number\gets\ fileReadInt(fileNo);\\
      numbers[nNumbers]\gets\ number;\\
      nNumbers\gets\ nNumbers+1;\\
    \END\\
  \END\\
  \textbf{catch}\ (error)\BEGIN
    \textbf{throw}\ ;\\
  \END\\
  \textbf{finally} \BEGIN
    fileClose(fileNo);\\
  \END\\
  \RETURN{nNumbers};\\
\ENDPROCEDURE
\end{pseudocode}

\end{document}
