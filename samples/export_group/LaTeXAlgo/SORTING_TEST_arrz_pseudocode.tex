\documentclass[a4paper,10pt]{article}

\usepackage{pseudocode}
\usepackage{ngerman}
\usepackage{amsmath}

\DeclareMathOperator{\oprdiv}{div}
\DeclareMathOperator{\oprshl}{shl}
\DeclareMathOperator{\oprshr}{shr}
\title{Structorizer LaTeX pseudocode Export of SORTING\_TEST.arrz}
\author{Kay G"urtzig}
\date{09.06.2021}

\begin{document}

\begin{pseudocode}{bubbleSort}{values }
\label{bubbleSort}
\COMMENT{ Implements the well-known BubbleSort algorithm. }\\
\COMMENT{ Compares neigbouring elements and swaps them in case of an inversion. }\\
\COMMENT{ Repeats this while inversions have been found. After every }\\
\COMMENT{ loop passage at least one element (the largest one out of the }\\
\COMMENT{ processed subrange) finds its final place at the end of the }\\
\COMMENT{ subrange. }\\
\PROCEDURE{bubbleSort}{values}
  ende\gets\ length(values)-2\\
  \REPEAT
    \COMMENT{ The index of the most recent swapping (-1 means no swapping done). }\\
    posSwapped\gets-1\\
    \FOR i \gets 0 \TO ende  \DO
      \IF values[i]>values[i+1] \THEN
      \BEGIN
        temp\gets\ values[i]\\
        values[i]\gets\ values[i+1]\\
        values[i+1]\gets\ temp\\
        posSwapped\gets\ i\\
      \END\\
    ende\gets\ posSwapped-1\\
  \UNTIL posSwapped<0\\
\ENDPROCEDURE
\end{pseudocode}


\begin{pseudocode}{maxHeapify}{heap, i, range }
\label{maxHeapify}
\COMMENT{ Given a max-heap '{}heap\textasciiacute{} with element at index '{}i\textasciiacute{} possibly }\\
\COMMENT{ violating the heap property wrt. its subtree upto and including }\\
\COMMENT{ index range-1, restores heap property in the subtree at index i }\\
\COMMENT{ again. }\\
\PROCEDURE{maxHeapify}{heap, i, range}
  \COMMENT{ Indices of left and right child of node i }\\
  right\gets(i+1)*2\\
  left\gets\ right-1\\
  \COMMENT{ Index of the (local) maximum }\\
  max\gets\ i\\
  \IF left<range\ \AND\ heap[left]>heap[i] \THEN
    max\gets\ left\\
  \IF right<range\ \AND\ heap[right]>heap[max] \THEN
    max\gets\ right\\
  \IF max\neq\ i \THEN
  \BEGIN
    temp\gets\ heap[i]\\
    heap[i]\gets\ heap[max]\\
    heap[max]\gets\ temp\\
    \CALL{maxHeapify}{heap,max,range}\\
  \END\\
\ENDPROCEDURE
\end{pseudocode}


\begin{pseudocode}{partition}{values, start, stop, p }
\label{partition}
\COMMENT{ Partitions array '{}values\textasciiacute{} between indices '{}start\textasciiacute{} und '{}stop\textasciiacute{}-1 with }\\
\COMMENT{ respect to the pivot element initially at index '{}p\textasciiacute{} into smaller }\\
\COMMENT{ and greater elements. }\\
\COMMENT{ Returns the new (and final) index of the pivot element (which }\\
\COMMENT{ separates the sequence of smaller elements from the sequence }\\
\COMMENT{ of greater elements). }\\
\COMMENT{ This is not the most efficient algorithm (about half the swapping }\\
\COMMENT{ might still be avoided) but it is pretty clear. }\\
\PROCEDURE{partition}{values, start, stop, p}
  \COMMENT{ Cache the pivot element }\\
  pivot\gets\ values[p]\\
  \COMMENT{ Exchange the pivot element with the start element }\\
  values[p]\gets\ values[start]\\
  values[start]\gets\ pivot\\
  p\gets\ start\\
  \COMMENT{ Beginning and end of the remaining undiscovered range }\\
  start\gets\ start+1\\
  stop\gets\ stop-1\\
  \COMMENT{ Still unseen elements? }\\
  \COMMENT{ Loop invariants: }\\
  \COMMENT{ 1. p = start - 1 }\\
  \COMMENT{ 2. pivot = values[p] }\\
  \COMMENT{ 3. i \textless start \textrightarrow{} values[i] ≤ pivot }\\
  \COMMENT{ 4. stop \textless i \textrightarrow{} pivot \textless values[i] }\\
  \WHILE start\leq\ stop \DO
  \BEGIN
    \COMMENT{ Fetch the first element of the undiscovered area }\\
    seen\gets\ values[start]\\
    \COMMENT{ Does the checked element belong to the smaller area? }\\
    \IF seen\leq\ pivot \THEN
    \BEGIN
      \COMMENT{ Insert the seen element between smaller area and pivot element }\\
      values[p]\gets\ seen\\
      values[start]\gets\ pivot\\
      \COMMENT{ Shift the border between lower and undicovered area, }\\
      \COMMENT{ update pivot position. }\\
      p\gets\ p+1\\
      start\gets\ start+1\\
    \END\\
    \ELSE
    \BEGIN
      \COMMENT{ Insert the checked element between undiscovered and larger area }\\
      values[start]\gets\ values[stop]\\
      values[stop]\gets\ seen\\
      \COMMENT{ Shift the border between undiscovered and larger area }\\
      stop\gets\ stop-1\\
    \END\\
  \END\\
  \RETURN{p}\\
\ENDPROCEDURE
\end{pseudocode}


\begin{pseudocode}{testSorted}{numbers }
\label{testSorted}
\COMMENT{ Checks whether or not the passed-in array is (ascendingly) sorted. }\\
\PROCEDURE{testSorted}{numbers}
  isSorted\gets\TRUE\\
  i\gets\ 0\\
  \COMMENT{ As we compare with the following element, we must stop at the penultimate index }\\
  \WHILE isSorted\ \AND(i\leq\ length(numbers)-2) \DO
  \BEGIN
    \COMMENT{ Is there an inversion? }\\
    \IF numbers[i]>numbers[i+1] \THEN
      isSorted\gets\FALSE\\
    \ELSE
      i\gets\ i+1\\
  \END\\
  \RETURN{isSorted}\\
\ENDPROCEDURE
\end{pseudocode}


\begin{pseudocode}{buildMaxHeap}{heap }
\label{buildMaxHeap}
\COMMENT{ Runs through the array heap and converts it to a max-heap }\\
\COMMENT{ in a bottom-up manner, i.e. starts above the "{}leaf"{} level }\\
\COMMENT{ (index \textgreater= length(heap) div 2) and goes then up towards }\\
\COMMENT{ the root. }\\
\PROCEDURE{buildMaxHeap}{heap}
  lgth\gets\ length(heap)\\
  \FOR k \gets lgth\oprdiv\ 2-1 \DOWNTO 0  \DO
    \CALL{maxHeapify}{heap,k,lgth}\\
\ENDPROCEDURE
\end{pseudocode}


\begin{pseudocode}{quickSort}{values, start, stop }
\label{quickSort}
\COMMENT{ Recursively sorts a subrange of the given array '{}values\textasciiacute{}.  }\\
\COMMENT{ start is the first index of the subsequence to be sorted, }\\
\COMMENT{ stop is the index BEHIND the subsequence to be sorted. }\\
\PROCEDURE{quickSort}{values, start, stop}
  \COMMENT{ At least 2 elements? (Less don'{}t make sense.) }\\
  \IF stop\geq\ start+2 \THEN
  \BEGIN
    \COMMENT{ Select a pivot element, be p its index. }\\
    \COMMENT{ (here: randomly chosen element out of start ... stop-1) }\\
    p\gets\ random(stop-start)+start\\
    \COMMENT{ Partition the array into smaller and greater elements }\\
    \COMMENT{ Get the resulting (and final) position of the pivot element }\\
    p\gets\CALL{partition}{values,start,stop,p}\\
    \COMMENT{ Sort subsequances separately and independently ... }\\
    \textbf{parallel}\BEGIN
      \textbf{thread}\ 1\BEGIN
        \COMMENT{ Sort left (lower) array part }\\
        \CALL{quickSort}{values,start,p}\\
      \END\\
      \textbf{thread}\ 2\BEGIN
        \COMMENT{ Sort right (higher) array part }\\
        \CALL{quickSort}{values,p+1,stop}\\
      \END\\
    \END\\
  \END\\
\ENDPROCEDURE
\end{pseudocode}


\begin{pseudocode}{heapSort}{values }
\label{heapSort}
\COMMENT{ Sorts the array '{}values\textasciiacute{} of numbers according to he heap sort }\\
\COMMENT{ algorithm }\\
\PROCEDURE{heapSort}{values}
  \CALL{buildMaxHeap}{values}\\
  heapRange\gets\ length(values)\\
  \FOR k \gets heapRange-1 \DOWNTO 1  \DO
  \BEGIN
    heapRange\gets\ heapRange-1\\
    \COMMENT{ Swap the maximum value (root of the heap) to the heap end }\\
    maximum\gets\ values[0]\\
    values[0]\gets\ values[heapRange]\\
    values[heapRange]\gets\ maximum\\
    \CALL{maxHeapify}{values,0,heapRange}\\
  \END\\
\ENDPROCEDURE
\end{pseudocode}


\begin{pseudocode}{SORTING\_TEST\_MAIN}{ }
\label{SORTING\_TEST\_MAIN}
\COMMENT{ Creates three equal arrays of numbers and has them sorted with different sorting algorithms }\\
\COMMENT{ to allow performance comparison via execution counting ("{}Collect Runtime Data"{} should }\\
\COMMENT{ sensibly be switched on). }\\
\COMMENT{ Requested input data are: Number of elements (size) and filing mode. }\\
\MAIN
  \REPEAT
    \)input\(elementCount\\
  \UNTIL elementCount\geq\ 1\\
  \REPEAT
    \)input\(\)"{}Filling:\ 1\ =\ random,\ 2\ =\ increasing,\ 3\ =\ decreasing"{}\(,modus\\
  \UNTIL modus=1\ \OR\ modus=2\ \OR\ modus=3\\
  \FOR i \gets 0 \TO elementCount-1  \DO
  \BEGIN
    \IF modus=1 \THEN
      values1[i]\gets\ random(10000)\\
    \ELSEIF modus=2 \THEN
      values1[i]\gets\ i\\
    \ELSEIF modus=3 \THEN
      values1[i]\gets-i\\
  \END\\
  \COMMENT{ Copy the array for exact comparability }\\
  \FOR i \gets 0 \TO elementCount-1  \DO
  \BEGIN
    values2[i]\gets\ values1[i]\\
    values3[i]\gets\ values1[i]\\
  \END\\
  \textbf{parallel}\BEGIN
    \textbf{thread}\ 1\BEGIN
      \CALL{bubbleSort}{values1}\\
    \END\\
    \textbf{thread}\ 2\BEGIN
      \CALL{quickSort}{values2,0,elementCount}\\
    \END\\
    \textbf{thread}\ 3\BEGIN
      \CALL{heapSort}{values3}\\
    \END\\
  \END\\
  ok1\gets\CALL{testSorted}{values1}\\
  ok2\gets\CALL{testSorted}{values2}\\
  ok3\gets\CALL{testSorted}{values3}\\
  \IF \NOT\ ok1\ \OR\ \NOT\ ok2\ \OR\ \NOT\ ok3 \THEN
    \FOR i \gets 0 \TO elementCount-1  \DO
      \IF values1[i]\neq\ values2[i]\ \OR\ values1[i]\neq\ values3[i] \THEN
        \OUTPUT{\)"{}Difference\ at\ ["{}\(,i,\)"{}]:\ "{}\(,values1[i],\)"{}\ \textless-\textgreater\ "{}\(,values2[i],\)"{}\ \textless-\textgreater\ "{}\(,values3[i]}\\
  \REPEAT
    \)input\(\)"{}Show\ arrays\ (yes/no)?"{}\(,show\\
  \UNTIL show=\)"{}yes"{}\(\ \OR\ show=\)"{}no"{}\(\\
  \IF show=\)"{}yes"{}\( \THEN
    \FOR i \gets 0 \TO elementCount-1  \DO
      \OUTPUT{\)"{}["{}\(,i,\)"{}]:\textbackslash{}t"{}\(,values1[i],\)"{}\textbackslash{}t"{}\(,values2[i],\)"{}\textbackslash{}t"{}\(,values3[i]}\\
\ENDMAIN
\end{pseudocode}

\end{document}
